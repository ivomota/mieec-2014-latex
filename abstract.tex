\chapter*{Resumo}
%\addcontentsline{toc}{chapter}{Resumo}

O Twitter é uma das redes sociais atuais que mais informação gera todos os dias. Face à sua dimensão, foi desenvolvido o TweeProfiles, uma ferramenta que recorre a esta rede social, mais concretamente às mensagens partilhadas neste serviço. Esta ferramenta utiliza técnicas de \textit{Data Mining} para identificar padrões, apresentados através de \textit{clusters} de Tweets, em que são analisados, o conteúdo na forma de texto, as ligações sociais, e as dimensões espaço-temporais das mensagens.

Face ao aumento do número de utilizadores que recorrem a smartphones para acederem ao Twitter, o número de fotografias partilhadas neste serviço tem crescido significativamente nos últimos anos. Esta dissertação teve como objetivo principal o desenvolvimento de uma extensão da ferramenta TweeProfiles, através de técnicas de visão por computador e \textit{Data Mining}, que permita a identificação de padrões espaço-temporais através da informação das imagens partilhadas no serviço de \textit{microblogging} Twitter. Para a sua concretização foi desenvolvido um módulo que utiliza o conceito de vocabulário visual para a representação das imagens de uma forma mais compacta e eficiente. 


Os resultados obtidos podem ser visualizados através de uma aplicação web que permite a navegação e visualização pelas imagens e dimensões espaço-temporais dos \textit{clusters}.

\chapter*{Abstract}
%\addcontentsline{toc}{chapter}{Abstract}

Twitter is one of social networks that generates more information every day. Due to it is dimension, was developed a tool called TweeProfiles, which uses this social network, more specifically the messages shared in this service. This tool uses data mining techniques to identify patterns presented through clusters of Tweets, each of them are analyzed in their respective: content as text, social connections, and dimensions of spatial and temporal messages.

Given the increasing number of users who use smartphones to access Twitter, the number of shared photos in this service has grown significantly in recent years. This thesis had the main objective of developing an extension of TweeProfiles tool. Through techniques of computer vision and data mining, this tool allow the identification of spatiotemporal patterns using all information in the shared images in the microblogging service Twitter. For it is implementation, a module was developed using the concept of visual vocabulary for representing images in a more compact and efficient way.

The results can be visualized through a web application that allows browsing and viewing the images and spatial and temporal dimensions of clusters. 
