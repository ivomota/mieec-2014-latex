\chapter*{Resumo}
%\addcontentsline{toc}{chapter}{Resumo}

O Twitter é uma das redes sociais atuais que mais informação gera todos os dias. Face à sua dimensão, foi desenvolvido o TweeProfiles, uma ferramenta que analisa as mensagens partilhadas neste serviço. Esta ferramenta utiliza técnicas de \textit{Data Mining} para identificar padrões, apresentados através de \textit{clusters} de Tweets, em que são analisados, o conteúdo na forma de texto, as ligações sociais, e as dimensões espaço-temporais das mensagens.

Face ao aumento do número de utilizadores que recorrem a smartphones para acederem ao Twitter, o número de fotografias partilhadas neste serviço tem crescido significativamente nos últimos anos. Esta dissertação teve como objetivo principal o desenvolvimento de uma extensão da ferramenta TweeProfiles, através de técnicas de processamento de imagem e \textit{data mining}, que permita a identificação de padrões espaço-temporais através da informação das imagens partilhadas no serviço de \textit{microblogging} Twitter. Para a sua concretização foi desenvolvido um módulo que utiliza o conceito de vocabulário visual para a representação das imagens de uma forma mais compacta e eficiente. 


Os resultados obtidos podem ser visualizados através de uma aplicação web que permite a navegação e visualização pelas imagens e dimensões espaço-temporais dos \textit{clusters}.

\chapter*{Abstract}
%\addcontentsline{toc}{chapter}{Abstract}

Twitter is one of social networks that generates more information on a continuous basic. Due to its dimension, a tool called TweeProfiles was created, which uses the messanges posted in this social network. This tool uses data mining techniques to identify patterns presented as clusters of Tweets, according to four dimensions: textual,  content, social connections, spatial and temporal characteristics.

Given the increasing number of users who use smartphones to access Twitter, the number of shared photos in this service has grown significantly in recent years. The main goal of this thesis is developing an extension of the TweeProfiles tool that also looks for patterns in those images. Through techniques of computer vision and data mining, this tool enables the identification of spatio-temporal patterns using all information in the shared images. The implementation is based on the concept of visual vocabulary for representing images in a more compact and efficient way.

The results can be visualized through a web application that allows browsing and viewing the images and spatial and temporal dimensions of clusters. 
