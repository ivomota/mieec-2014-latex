\chapter{Módulo da informação visual} \label{chap:chap3}

Neste capítulo será introduzido o módulo da informação visual desenvolvido, sendo apresentado a estrutura e organização deste sistema. Como referido no capítulo~\ref{chap:intro}, este projeto pretende estender a ferramenta TweeProfiles descrita na secção~\ref{sec:tweep}, dando-lhe uma dimensão de conteúdo diferente à que possuí, em que a informação tido em conta, em conjunto com a espacial e temporal, passa a ser imagens partilhadas em \textit{tweets} e não o texto. Para que isto seja realizável, foi necessário o desenvolvimento de um módulo que efetue a recolha os dados com a devida filtragem, extraia e processe a informação visual e armazene essa informação de modo a que fosse possível a sua integração com o TweeProfiles.

\section{Recolha dos dados}

\subsection{Descrição dos dados}

O primeiro passo para a realização deste projeto de dissertação foi a recolha dos dados necessários. Estes dados foram recolhidos através de uma base de dados mongodb previamente criada usando a plataforma Socialbus, anteriormente designada por TwitterEcho~\cite{Boanjak2012}. Este dados vêm sobre a forma de objetos JSON e possuem informação relativa a cada tweet como pode ser visto no anexo~\ref{ap1}. Estes objetos encontram-se todos num só documento mongodb que se caracteriza pelas características apresentadas na tabela~\ref{tab:nbrtweets}

\vspace{5 mm}
\begin{table}[h]
\centering
\begin{tabular}{|l|c|c|l|l|l|}
\hline
          & Total                        & Com imagem & Twitter                  & TwitPic                   & Instagram                  \\ \hline
Nº Tweets & \multicolumn{1}{r|}{1704273} & 86349      & \multicolumn{1}{c|}{202} & \multicolumn{1}{c|}{6100} & \multicolumn{1}{c|}{79210} \\ \hline
\end{tabular}
\caption{Descrição em números do total de tweets com indicação, nos que contém URL para imagem, do número de tweets por serviço de partilha de imagem}
\label{tab:nbrtweets}
\end{table}
\vspace{5 mm}

Estes tweets foram recolhidos entre o dia 17 e 19 de Junho de 2013 com conteúdo partilhado somente contendo texto escrito em português do Brasil, tendo sido isto foi possível graças á capacidade de filtragem da ferramenta Socialbus~\cite{Boanjak2012} já referida anteriormente. Estas datas coincidiram com um evento ocorrido no Brasil, mais especificamente, as manifestações do ano passado do povo brasileiro contra o seu governo. Este foi um dos motivos da escolha desta base de dados, pois apresentava tweets que poderiam ser interessante para encontrar padrões ou eventos através das imagens partilhadas pelos brasileiros nas ruas, aliadas sempre às dimensões espaço-temporais. 

\subsection{Filtragem dos dados}

Após estar definido o conjunto de dados a utilizar, foi necessário realizar uma filtragem dos dados de modo a apresentem a informação necessária para a realização deste projeto. O primeiro passo desta filtragem foi recolher todos os tweets que contivessem no seu objeto um URL para uma imagem, sendo que esse URl teria de pertencer a um dos seguintes serviços: 

\begin{itemize}
\item Twitter
\item TwitPic
\item Instagram
\end{itemize}

e teria que esse URl ser válio, isto é, foi feita uma prévia verificação se a imagem estaria ainda disponível através do endereço existente.

Para ser mais fácil posteriormente uma seleção mais cuidadosa dos tweets foi criada uma base de dados local (SQLite) com a seguinte tabela:

\begin{lstlisting}[language=SQL]
create table if not exists IMAGENS ( 
	id integer PRIMARY KEY AUTOINCREMENT, 
	id_tweet text, 
	servico text, 
	url text, 
	tipo text, 
	retweet text 
); 
\end{lstlisting}

Em que se descreve cada coluna da seguinte forma:

\begin{description}
\item[id] - id da linha da tabela;
\item[id\_tweet] - id do tweet na base de dados Mongodb;
\item[servico] - nome do serviço de alojamento da imagem;
\item[url] - endereço url para a imagem fonte;
\item[tipo] - este atributo identifica se a imagem pertence a um tweet ou retweet;
\item[retweet] - caso a imagem pertença a um retweet, este atributo pode assumir o valor "primeiro" no caso de ser o primeiro retweet, do tweet original, na base de dados mongodb, ou caso contrário, assume o valor NULL.
\end{description}

Isto permitiu realizar de uma forma rápida alguns teste no download de algumas imagens pelos diferentes serviços, para além de permitir fazer uma seleção fácil e rápida de tweets, retweets ou por exemplo, do primeiro retweet no caso de existir vários retweets de um determinado tweet. 

Após a criação desta base de dados, ficou decidido selecionar todos os objetos da base de dados Mongodb do tipo tweet em que o seu serviço fosse o Instagram. Esta decisão deveu-se ao facto do Instagram apresentar um número maior de imagens 

\subsection{Conjunto de dados final}

Por fim armazenou-se um ficheiro JSON com todos os dados a ser utilizados e foi realizado o download de todas as imagens relativas a cada tweet e armazenadas localmente em formato JPEG, que se apresentava como formato de origem das imagens descarregadas. 

Relativamente aos objetos de cada tweet presentes no ficheiro JSON, optou-se por não armazenar todas as instâncias para reduzir o tamanho do ficheiro, tendo sido apenas incluídas as representadas no seguinte exemplo de um objeto de um tweet:

\lstinputlisting[language=Python]{./code/tweets_instagram_exemple.json}



%\section{Extração e processamento da informação visual}
%
%
%\section{Armazenamento da informação visual}