%% FEUP THESIS STYLE for LaTeX2e
%% how to use feupteses (portuguese version)
%%
%% FEUP, JCL & JCF, 31 Jul 2012
%%
%% PLEASE send improvements to jlopes at fe.up.pt and to jcf at fe.up.pt
%%
%%========================================
%% Commands: pdflatex tese
%%           bibtex tese
%%           makeindex tese (only if creating an index) 
%%           pdflatex tese
%% Alternative:
%%          latexmk -pdf texe.tex
%%========================================

\documentclass[11pt,a4paper,twoside,openright]{report}

%% For iso-8859-1 (latin1), comment next line and uncomment the second line
\usepackage[utf8]{inputenc}
%\usepackage[latin1]{inputenc}


%% Portuguese version

%% MIEEC options
%\usepackage[portugues,mieec]{feupteses}
\usepackage[portugues,mieec,juri]{feupteses}
%\usepackage[portugues,mieec,final]{feupteses}
%\usepackage[portugues,mieec,final,onpaper]{feupteses}

%%Tables
\usepackage{multirow}
\usepackage[table]{xcolor}

%% Gantt Diagram
\usepackage{gantt}

\usepackage{graphicx}
\usepackage{caption}
\usepackage{subcaption}

% %pseudo-code
\usepackage{amsmath}
\usepackage{algorithm}
\usepackage{algpseudocode}

\makeatletter
\def\BState{\State\hskip-\ALG@thistlm}
\newcommand{\newalgname}[1]{%
  \renewcommand{\ALG@name}{#1}%
}
\newalgname{Algoritmo}% All algorithms will be called "Algoritmo"
\makeatother


%% package for code
\usepackage{listings}

\usepackage{color}
 
\definecolor{codegreen}{rgb}{0,0.6,0}
\definecolor{codegray}{rgb}{0.5,0.5,0.5}
\definecolor{codepurple}{rgb}{0.58,0,0.82}
\definecolor{backcolour}{rgb}{0.95,0.95,0.92}
 
\lstdefinestyle{mystyle}{
    backgroundcolor=\color{backcolour},   
    commentstyle=\color{codegreen},
    keywordstyle=\color{magenta},
    numberstyle=\tiny\color{codegray},
    stringstyle=\color{codepurple},
    basicstyle=\footnotesize,
    breakatwhitespace=false,         
    breaklines=true,                 
    captionpos=b,                    
    keepspaces=true,                 
    numbers=left,                    
    numbersep=5pt,                  
    showspaces=false,                
    showstringspaces=false,
    showtabs=false,                  
    tabsize=2
}
 
\lstset{style=mystyle}

%% For other degrees
% \usepackage[portugues]{feupteses} % you must define the degree bellow

%% Options: 
%% - portugues: titles, etc in portuguese
%% - onpaper: links are not shown (for paper versions)
%% - backrefs: include back references from bibliography to citation place

%% Uncomment to create an index (at the end of the document)
%\makeindex

%% Path to the figures directory
%% TIP: use folder ``figures'' to keep all your figures
\graphicspath{{figures/}}


%%----------------------------------------
%% TIP: if you want to define more macros, use an external file to keep them
\include{mymacros}
%%----------------------------------------

%%========================================
%% Start of document
%%========================================
\begin{document}


%%----------------------------------------
%% Information about the work
%%----------------------------------------
\title{Olhó-passarinho: uma extensão do TweeProfiles para fotografias}
\author{Ivo Filipe Valente Mota}


%% Uncomment next line for date of submission
%\thesisdate{30 de Junho de 2014}

%% Comment next line for copyright text if not used
\copyrightnotice{Ivo Mota, 2014}

\supervisor{Orientador}{Luís Filipe Pinto de Almeida Teixeira (PhD)}

%% Uncomment next line if necessary
\supervisor{Co-orientador}{Carlos Manuel Milheiro de Oliveira Pinto Soares (PhD)}

%% Uncomment committee stuff in the final version if used
%\committeetext{Aprovado em provas públicas pelo Júri:}
%\committeemember{Presidente}{Nome do presidente do júri}
%\committeemember{Arguente}{Nome do arguente do júri}
%\committeemember{Vogal}{Nome do vogal do júri}
%\signature


%% Specify cover logo (in folder ``figures'')
\logo{uporto-feup.pdf}
 
%% Uncomment next line for additional text  below the author's name (front page)
%\additionalfronttext{Preparação da Dissertação}

%%----------------------------------------
%% Preliminary materials
%%----------------------------------------

% remove unnecessary \include{} commands
\begin{Prolog}
  \chapter*{Resumo}
%\addcontentsline{toc}{chapter}{Resumo}

O Twitter é uma das redes sociais atuais que mais informação gera todos os dias. Face à sua dimensão, foi desenvolvido o TweeProfiles, uma ferramenta que analisa as mensagens partilhadas neste serviço. Esta ferramenta utiliza técnicas de \textit{Data Mining} para identificar padrões, apresentados através de \textit{clusters} de Tweets, em que são analisados, o conteúdo na forma de texto, as ligações sociais, e as dimensões espaço-temporais das mensagens.

Face ao aumento do número de utilizadores que recorrem a smartphones para acederem ao Twitter, o número de fotografias partilhadas neste serviço tem crescido significativamente nos últimos anos. Esta dissertação teve como objetivo principal o desenvolvimento de uma extensão da ferramenta TweeProfiles, através de técnicas de processamento de imagem e \textit{data mining}, que permita a identificação de padrões espaço-temporais através da informação das imagens partilhadas no serviço de \textit{microblogging} Twitter. Para a sua concretização foi desenvolvido um módulo que utiliza o conceito de vocabulário visual para a representação das imagens de uma forma mais compacta e eficiente. 


Os resultados obtidos podem ser visualizados através de uma aplicação web que permite a navegação e visualização pelas imagens e dimensões espaço-temporais dos \textit{clusters}.

\chapter*{Abstract}
%\addcontentsline{toc}{chapter}{Abstract}

Twitter is one of social networks that generates more information on a continuous basic. Due to its dimension, a tool called TweeProfiles was created, which uses the messanges posted in this social network. This tool uses data mining techniques to identify patterns presented as clusters of Tweets, according to four dimensions: textual,  content, social connections, spatial and temporal characteristics.

Given the increasing number of users who use smartphones to access Twitter, the number of shared photos in this service has grown significantly in recent years. The main goal of this thesis is developing an extension of the TweeProfiles tool that also looks for patterns in those images. Through techniques of computer vision and data mining, this tool enables the identification of spatio-temporal patterns using all information in the shared images. The implementation is based on the concept of visual vocabulary for representing images in a more compact and efficient way.

The results can be visualized through a web application that allows browsing and viewing the images and spatial and temporal dimensions of clusters. 
 % the abstract
  \chapter*{Agradecimentos}
\addcontentsline{toc}{chapter}{Agradecimentos}

Em primeiro lugar quero deixar os meus agradecimentos aos meus orientadores, Professor Doutor 


\vspace{10mm}
\flushleft{Ivo Mota}

\vfill

This work is partially funded by National Funds through the FCT – Fundação para a Ciência e a Tecnologia (Portuguese Foundation for Science and Technology) within projects "REACTION (UTAustin/EST-MAI/0006/2009)" and "POPSTAR (PTDC/CPJ-CPO/116888/2010)" as well as Project "NORTE-07-0124-FEDER-000059", which is funded by the North Portugal Regional Op- erational Programme (ON.2 – O Novo Norte), under the National Strategic Reference Framework (NSRF), through the European Regional Development Fund (ERDF), and the Portuguese funding agency, Fundação para a Ciência e a Tecnologia (FCT).   % the acknowledgments
  \cleardoublepage
\thispagestyle{plain}

\vspace*{8cm}

\begin{flushright}
   \textsl{''Logic will get you from A to Z.\\
           Imagination will get you everywhere.''} \\
\vspace*{1.5cm}
           Albert Einstein
\end{flushright}
    % initial quotation if desired
  \cleardoublepage
  \pdfbookmark[0]{Conteúdo}{contents}
  \tableofcontents
  \cleardoublepage
  \pdfbookmark[0]{Lista de Figuras}{figures}
  \listoffigures
  \cleardoublepage
  \pdfbookmark[0]{Lista de Tabelas}{tables}
  \listoftables
  \chapter*{Abreviaturas e Símbolos}
%\addcontentsline{toc}{chapter}{Abbreviations}
\chaptermark{ABREVIATURAS E SÍMBOLOS}

\begin{flushleft}
\begin{tabular}{l p{0.8\linewidth}}
BoW		 & \textit{Bag Of Words}\\
DBSCAN   & \textit{Density-Based Spatial Clustering of Applications with Noise}\\
DoG      & \textit{Diference of Gaussian}\\
GLOH     & \textit{Gradient Location and Orientation Histogram}\\
HOG      & \textit{Histogram of Oriented Gradients}\\
MS		 & \textit{Maximally Stable}\\
SA		 & \textit{Shape Adapted}\\
SIFT     & \textit{Scale-Invariant Feature Transform}\\
SURF     & \textit{Speeded Up Robust Features}



%UNCOL    & UNiversal COmpiler-oriented Language\\
%Loren    & Lorem ipsum dolor sit amet, consectetuer adipiscing
%elit. Sed vehicula lorem commodo dui\\
%WWW      & \emph{World Wide Web}
\end{tabular}
\end{flushleft}

  % the list of abbreviations used
\end{Prolog}

%%----------------------------------------
%% Body\\

%%----------------------------------------

\StartBody

%% TIP: use a separate file for each chapter
\chapter{Introdução} \label{chap:intro}

Neste capítulo é feita uma introdução ao projeto desenvolvido no âmbito da dissertação com a apresentação do seu contexto, a motivação para o seu desenvolvimento, os objetivos a alcançar e a descrição da estrutura deste documento.

\section{Contexto}

As redes sociais são uma excelente fonte de informação sempre em atualização, que fornecem aos investigadores uma vasta quantidade e variedade de dados. Este dados apresentam-se de diferentes formas como texto, imagem ou mesmo  vídeos. Esta informação está acessível através de API's disponibilizadas pelos próprios serviços, e pode ser assim utilizada para, por exemplo, realizar análise de sentimentos ou opiniões partilhadas através de texto por utilizadores da rede social Twitter~\cite{Pak2010, twitter}, ou mesmo para a descoberta de novas técnicas mais eficazes na pesquisa de imagens no serviço Flickr~\cite{flickr} utilizando anotações inseridas por utilizadores~\cite{Li2008}.  

O Twitter faz parte do grupo de redes sociais existentes que mais informação produz todos os dias, sendo caracterizado como um serviço de microblogging, que permite aos utilizadores partilharem mensagens, designadas por tweets, até um máximo de 140 caracteres. Essas mensagens podem conter, para além de texto, imagens ou links para imagens de outros serviços, como por exemplo, o Instagram~\cite{instagram} ou o Twitpic~\cite{twitpic}. Ao contrário do que acontece com outras redes sociais como o Facebook~\cite{facebook} e Linkedin~\cite{linkedin} que utilizam uma rede de comunicação bi-direcional, o Twitter utiliza uma infraestrutura assimétrica onde existem \textit{"friends"} e \textit{"followers"}. Os \textit{"friends"} correspondem às contas das pessoas que o utilizador segue e os \textit{"followers"} às contas das pessoas que o seguem~\cite{Russell2011}.

O TweeProfiles~\cite{Cunha2013}, é uma ferramenta que tem como principal objetivo identificar padrões em mensagens escritas, partilhadas na rede social Twitter. Esta ferramenta utiliza técnicas de \textit{data mining}, mais precisamente de \textit{text mining}. A principal característica do TweeProfiles é o facto de utilizar a tarefa de \textit{clustering} para identificar padrões em mensagens partilhadas no Twitter, através do conteúdo das mensagens (o texto) e das dimensões espaço-temporais das mesmas. 

\section{Motivação} \label{sec:motiv}

Devido ao grande número de utilizadores e de informação partilhada a todo o instante no Twitter, este torna-se um excelente serviço de recolha de dados, proporcionando aos investigadores e empresas uma quantidade e variedade de dados necessários para o desenvolvimento de ferramentas de análise de dados e extração de conhecimento.

As mensagens partilhadas no Twitter sobre a forma de texto têm sido uma das grandes fontes de dados utilizadas por muitas ferramentas como o TweeProfiles~\citet{Cunha2013}. No entanto, apesar de se tratar de uma rede social em que a maioria da informação disponível se encontra em forma de texto, o Twitter também permite a partilha de imagens a partir do seu próprio serviço, ou através de outros serviços como Twitpic ou Instagram. Estas imagens também podem ser utilizadas para a análise e extração de conhecimento, pois o seu conteúdo pode mesmo em muitos casos complementar o texto ou até mesmo, o substituir.

A análise de informação visual é assim um acréscimo importante para o desenvolvimento de ferramentas de extração de conhecimento das redes sociais. 

\section{Objetivos} \label{sec:object}

Esta dissertação tem como principal objetivo a criação de uma extensão para o TweeProfiles através de técnicas de processamento de imagem e \textit{data mining}, que permita a identificação de padrões em imagens partilhadas no serviço de microblogging Twitter, através da identificação de \textit{clusters}.

Será assim necessário realizar a recolha dos dados alojados numa base de dados MongoDB~\cite{mongodb} criada através da plataforma Socialbus (anteriormente designada por TwitterEcho~\cite{Boanjak2012}). Esta plataforma consiste num projeto open source de desenvolvimento de uma ferramenta para extrair e armazenar tweets de uma determinada comunidade de utilizadores. Foi desenvolvido com o intuito de ajudar os investigadores a terem facilidade de acesso a uma base de dados de redes sociais, na sua maioria. Após recolhidos os dados será necessário o desenvolvimento de um módulo responsável pela recolha das imagens através do \textit{URL} existente nos tweets, do processamento da informação visual de modo a torná-la mais compacta e eficiente, e do armazenamento dessa informação. Por fim, a informação visual deverá ser integrada na ferramenta TweeProfiles, com objetivo de realizar o processo de \textit{Data Mining}, mais especificamente a tarefa de \textit{clustering} e desenvolver a aplicação para visualizar os \textit{clusters} nas diferentes dimensões. 

\section{Estrutura do documento} \label{sec:struct}

Este documento está organizado da seguinte forma: o Capítulo~\ref{chap:estarte} descreve conceitos e trabalhos relacionados e apresentada uma pesquisa sobre os vários domínios científicos necessários para o desenvolvimento deste projeto de dissertação. No Capítulo~\ref{chap:chap3} é apresentado o modelo desenvolvido para a extração, processamento e armazenamento da informação visual. Já no Capítulo~\ref{chap:chap4} é descrita a ferramenta Olhó-passarinho e a integração do módulo desenvolvido para a informação visual com a ferramenta TweeProfiles. Para finalizar é apresentado o Capítulo~\ref{chap:concl} com um resumo do trabalho desenvolvido e discute o desenvolvimento deste projeto de dissertação com sugestões de trabalho futuro a realizar.
%Devo melhorar
 

\chapter{Conceitos e Trabalhos Relacionados} \label{chap:estarte}

Neste capítulo é apresentado o estudo realizado, tendo em vista a aquisição de competências e conhecimentos necessários para o desenvolvimento do projeto, que se focam essencialmente em análise de métodos de \textit{data mining} e em técnicas aplicadas em visão por computador. Em primeiro ligar será exposto conteúdo relativamente a métodos de \textit{clustering} como uma tarefa de \textit{data mining}. Em seguida serão apresentadas formas de representação de imagens. Por fim, são referenciados alguns trabalhos relacionados com o projeto a desenvolver nesta dissertação.

\section{Clustering} \label{sec:cluster}

Extração de conhecimento em base de dados ou \textit{Data Mining} é um processo de exploração de grandes quantidades de dados que procura encontrar padrões "interessantes"~\citet{Han2006}. Trata-te assim de uma fusão de estatística aplicada, sistemas de lógica, inteligência artificial, \textit{Machine Learning} e gestão de base de dados~\citet{North2012}. Este processo é caracterizado por várias tarefas possíveis de ser aplicadas, dependendo do problema abordado, tais como~\citep{Fayyad1996}:

\begin{itemize}
\item Deteção de anomalias (outliers/ alterações/ desvios) - Identifica registos de dados incomuns, podendo ser erros nos dados ou objetos interessantes que apresentam comportamento diferente dos restantes;
\item Regras de associação - Procura relações entre variáveis que ocorram frequentemente;
\item Classificação - É a tarefa de generalizar uma estrutura conhecida e aplicar a novos dados, sendo essencialmente utilizada em tarefas de previsão;
\item Regressão - Tenta encontrar uma função que modela os dados com o mínimo de erro;
\item Resumo - Trata-se da representação mais compacta do conjunto de dados, que pode incluir visualização e descrição através de um relatório,
\item \textit{Clustering} - Tarefa de descobrir grupos em que os dados apresentam de alguma forma semelhanças, sem o uso de estruturas previamente conhecidas
\end{itemize}

Este projeto terá como uma das principais tarefas a realização \textit{clustering} sobre dados recolhidos e tratados de fotografias partilhadas no Twitter. Pode-se definir \textit{clustering} como "um processo de agrupamento de um conjunto de objetos de dados em vários grupos ou \textit{clusters}, de modo que os objetos dentro de um \textit{cluster} apresentem alta similaridade, mas que sejam muito diferentes de objetos de outros \textit{clusters}. Diferenças e semelhanças são avaliados com base nos valores de atributos que descrevem os objetos e muitas vezes envolvem medidas de distância"~\citet{Han2006}.

A realização de \textit{clustering} é assim uma escolha lógica para a extração de padrões em dados não supervisionados e para o agrupamento de \textit{tweets} pela sua semelhança em conteúdo, neste caso as imagens, e com a integração de outras dimensões, como o tempo e espaço.

O \textit{clustering} faz parte de um conjunto de técnicas aplicadas na aprendizagem não supervisionada. Enquanto que na aprendizagem supervisionada existe um conjunto de dados previamente analisados e rotulados que são usados para treinar um modelo capaz de encontrar relação entre os atributos desses dados com novos conjuntos de dados, na aprendizagem não supervisionada, não são utilizados conjuntos de dados previamente analisados e rotulados. Assim o processo de descoberta de padrões nos dados apenas tem em conta os dados presentes, tentando organizar as instâncias em grupos semelhantes~\citet{Liu2011}.

Nesta secção serão apresentadas as principais características e técnicas para aplicação da tarefa de \textit{clustering}, tais como, \textit{clustering} por partição, \textit{clustering} hierárquico, \textit{clustering} baseado em densidade, \textit{clustering} baseado em grelhas, funções de distância para o cálculo da similaridade entre objetos e por fim, a avaliação de \textit{clusters}.


\subsection{Clustering por Partição} \label{subsec:parti}

A utilização de métodos baseados em partições é a forma mais simples e elementar de realizar análise por \textit{clustering}, em que um conjunto de objetos é distribuído em vários grupos ou \textit{clusters} mais pequenos. É assumido que o número de \textit{clusters} é conhecido antes da realização da tarefa, sendo esse valor tomado como o ponto de partida para aplicação de métodos baseados em partição~\citet{Han2006}. 

O algoritmo \textit{k-means} é o melhor algoritmo de \textit{clustering} por partição e o mais utilizado devido à sua simplicidade e eficiência~\citet{Liu2011}. É apresentado como  sendo  um algoritmo de \textit{clustering} por partição, pois este divide o conjunto de dados em partições mais pequenas, formando assim os \textit{clusters}.

Inicialmente é necessário que o utilizador indique o valor de \textit{k} e o algoritmo irá iterativamente dividir o conjunto de objetos em \textit{k-clusters} diferentes, baseado em funções de distância~\citet{Liu2011} que são apresentadas na secção~\ref{subsec:dist}.

Cada \textit{cluster} apresenta um centroide que é o representante do grupo, sendo o valor médio de todos os objetos (instâncias) pertencentes ao \textit{cluster}. Este centroide é recalculado de forma iterativa até que seja atingido o critério de paragem. A convergência ou critério de paragem pode ser um dos seguintes enumerados:

\begin{enumerate}
\item Não ocorre (ou ocorre um valor mínimo) de alterações dos objetos para diferentes \textit{clusters}.
\item Não ocorre (ou ocorre um valor mínimo) de alterações dos centroides.
\item Diminuição mínima da \textbf{soma do erro quadrático} (SEQ),
\end{enumerate}

\begin{equation}
SEQ = \sum_{j=1}^{k} \sum_{x \in C_{j} } dist(x, m_{j})^{2} ,
\end{equation}

onde \textit{k} é o número de \textit{clusters} pretendidos, $ C_{j} $ é i-ésimo \textit{cluster}, $ m_{j} $ é o centroide do \textit{cluster} $ C_{j} $ e $ dist(x, m_{j}) $ é a distância entre uma instância \textbf{x} e o centroide $ m_{j} $.

Assim, "o algoritmo \textit{k-means} pode ser usado em qualquer aplicação com um conjunto de dados onde a média pode ser definida e calculada"~\citet{Liu2011}.

No \textbf{espaço euclidiano}, centroide de um \textit{cluster} é calculado da seguinte forma:

\begin{equation}
m_{j} =  \frac{1}{|C_{j}|} \sum_{x_{i} \in C_{j} }x_{i} ,
\end{equation}

onde $ |C_{j}| $ é o número de pontos (instâncias) no \textit{cluster} $ C_{j} $. A distância entre um ponto $ x_{i} $ a um centroide $ m_{j} $ é calculado da seguinte forma:

\begin{equation}
dist(x_{i}, m_{j}) = ||x_{i} - m_{j}|| =  \sqrt{(x_{i1} - m_{j1})^2 + (x_{i2} - m_{j2})^2 + ... + (x_{ir} - m_{jr})^2} .
\end{equation}

O pseudo-código deste algoritmo é apresentado no algoritmo~\ref{kmeans}.
% ESCREVER ALGORITMO

\begin{algorithm}
\caption{K-Means}\label{kmeans}
\begin{algorithmic}[1]
\Procedure{K-MEANS}{k: clusters, D: conjunto de dados}
	\State {escolher k objetos de D como centroides dos clusters iniciais;} 
	\Repeat 
		\State {(re) atribuir cada objeto a ao cluster ao qual o objeto é o mais similar;} 
		\State {actualizar o centroide do cluster;} 
	\Until{clusters sem alterações;}
	\State \textbf{return} {conjunto de k clusters;}
\EndProcedure 
\end{algorithmic}
\end{algorithm}

%\subsection{Representação de Clusters}

\subsection{Clustering Hierárquico} \label{subsec:hierar}

O \textit{clustering} hierárquico é outra abordagem importante na tarefa de \textit{clustering}. Os \textit{clusters} são criados sobre a forma de uma sequência em árvore (dendrograma). Os objetos (instâncias) encontram-se no fundo do diagrama, enquanto que o conjunto de todos os objetos encontra-se no topo do diagrama. Cada nó que se encontra no interior do diagrama possui nós filhos, sendo que cada nó representa um \textit{cluster}. Assim designam-se por \textit{clusters} irmãos, aqueles que derivam de um mesmo \textit{cluster}, isto é, do nó parente~\citet{Liu2011}. A Figura~\ref{fig:hierarquico} exemplifica a representação ilustrativa de um dendrograma onde o topo é representado o conjunto de todas as letras e o fim do cada letra individual. 

\begin{figure}[h]
\centering
\includegraphics[width=0.6\linewidth]{./figures/hierarquico}
\caption{Dendrograma ilustrativo da divisão entre nós do conjunto completo ao objeto individual no fundo~\citet{Bramer2007}}
\label{fig:hierarquico}
\end{figure}

Existem dois tipos principais de métodos de \textit{clustering} hierárquico, sendo eles~\citet{Liu2011} :

\begin{description}
\item[\textit{Clustering} por aglomeração:] O dendrograma é construído do nível mais baixo até ao mais alto, juntando sucessivamente e iterativamente os \textit{clusters} com maiores semelhanças até existir um único \textit{cluster} com todo o conjunto dos dados.

\item[\textit{Clustering} por divisão:] O dendograma é construído do nível mais alto até ao nível mais baixo, onde o processo tem início com um único \textit{cluster} que possui todos os objetos, sendo dividido sucessivamente em \textit{clusters} mais pequenos, até que estes sejam constituídos apenas por um único objeto.
\end{description}

Ao contrário do algoritmo \textit{k-means}, que apenas calcula a distância entre os centroides de cada grupo ou \textit{cluster}, no \textit{clustering} hierárquico podem ser usados os vários métodos apresentados em seguida para determinar a distância entre dois \textit{clusters}~\citet{Liu2011}: 

\begin{description}
\item[Método \textit{Single-Link}:] Neste método, a distância entre dois \textit{clusters} é determinada pela distância entre os dois objetos mais próximos (vizinhos mais próximo) pertencentes a \textit{clusters} diferentes.
\item[Método \textit{Complete-Link}:] Neste método, a distância entre dois \textit{clusters} é determinada pela maior distância entre dois objetos (vizinhos mais distante).
\item[Método Average-Link:] Este método tenta manter um compromisso entre a sensibilidade a \textit{outliers} do método \textit{Complete-Link} e a sensibilidade do método \textit{Single-Link} ao ruído existente nos dados. Para isso, é determinada a distância entre dois \textit{clusters} através da distância média entre todos os pares de objetos nos dois \textit{clusters}.
\item[Método Ward:] Este método, tenta minimizar a variância entre dois \textit{clusters} unidos.
\end{description}

Assim conclui-se que o \textit{clustering} hierárquico apresenta-se para determinados domínios, como bastante intuitivo para humanos, mas a interpretação dos resultados pode ser por vezes subjetiva. Outra característica interessante é o facto de, ao contrário do \textit{clustering} por partição, no \textit{clustering} hierárquico não ser necessário especificar logo à partida o número de \textit{clusters}. 

\subsection{Clustering Baseado em Densidade} \label{subsec:dbscan} 

Os métodos de ~\textit{clustering} por partição ou hierárquico estão preparados para encontrar \textit{clusters} que apresentam formas geométricas circulares, sendo ineficiente quando as formas destes grupos são por exemplo elípticas. Assim, para descobrir \textit{clusters} com formas arbitrárias, podem ser usados métodos baseados na densidade dos objetos~\cite{Han2006}. Um dos algoritmos mais conhecidos que utiliza este tipo de método é o DBSCAN (\textit{Density-Based Spatial Clustering of Applications with Noise}) que é capaz de encontrar \textit{clusters} através da análise da densidade e proximidade dos objetos pertencentes a um conjunto de dados. Para esta análise é necessário previamente atribuir um valor que definirá o raio da vizinhança considerada para cada objeto. Esse parâmetro é designado por $ \epsilon $ e terá de ser necessariamente maior que 0. Assim,  a $ \epsilon $-vizinhança de um objeto \textit{x} é o espaço dentro de um raio com valor $ \epsilon $, centrado em \textit{x}~\cite{Han2006}. Já para determinar a densidade de uma vizinhança, é utilizado o parâmetro \textit{MinPts} também previamente definido, que especifica o número mínimo de objetos vizinhos que um objeto necessita ter em seu redor, para ser considerado como objeto central.
Os passos necessários para a implementação deste método são apresentados no Algoritmo~\ref{dbscan}

\begin{algorithm}[ht]
\caption{DBSCAN}\label{dbscan}
\begin{algorithmic}[1]
\Procedure{DBSCAN}{\textit{MinPts}: limiar da vizinhança,  D: conjunto de dados, $ \varepsilon $ : raio  }
	\State {Marcar todos os objetos como não selecionados;} 
	\Repeat
		\State {escolher aleatóriamente um objeto \textit{p} não selecionado;} 
		\State {actualizar objeto \textit{p} como selecionado;} 
		\If {o $ \varepsilon$-vizinhança de \textit{p} tem pelo menos \textit{MinPts} objetos}
			\State {criar um novo \textit{cluster C} e adicionar objeto \textit{p} ao \textit{cluster C};}
			\State {Seja N o conjunto de objetos na $\varepsilon$-vizinhança de \textit{p}; }
			\For {cada ponto \textit{p'} em N}
				\If {\textit{p'} não selecionado}
					\State {Marcar \textit{p'} como selecionado}
					\If {a $\varepsilon$-vizinhança de \textit{p'} tem pelo menos \textit{MinPts}}
						\State {adicionar ponto a N}
					\EndIf
				\EndIf
				\If {\textit{p'} não pertence a nenhum cluster}
					\State {adicionar \textit{p'} a \textit{C}}
				\EndIf
			\EndFor
			\State {\textbf{output} C;}
		\Else { marcar \textit{p} como ruido}
		\EndIf
	\Until{todos os objetos selecionados;} 
\EndProcedure 
\end{algorithmic}
\end{algorithm}

Uma das grandes vantagens do \textit{clustering} baseado em densidade é que neste não é necessário uma prévia definição do número de \textit{clusters}, sendo apresentados os que encontrar consoante os dados que possui e os parâmetros definidos. 

% Notas: Falar talves do algoritmo OPTICS e DENCLUE

\subsection{Clustering Baseado em Grelhas} %por completar

Os métodos de \textit{clustering} discutidos até agora, apresentam algoritmos que se adaptam a distribuição dos dados no espaço. Em alternativa, o \textit{clustering} baseado em grelhas é orientado ao espaço, na medida em que divide o espaço em células independentemente da distribuição dos objetos de entrada. Este quantifica o espaço num número finito de células, que formam uma estrutura de grelhas sobre a qual são executadas as operações de \textit{clustering}. Este método apresenta como principal vantagem o tempo baixo de processamento, que normalmente é independente da quantidade de dados, no entanto, este depende do número de células em cada uma das dimensões no espaço quantizado~\cite{Han2006}.

O algoritmo STING (\textit{Statistical Information Grid})~\citet{Wang1997} é um dos algoritmos utilizados para \textit{clustering} baseado em grelhas. Este divide o espaço em células retangulares, correspondente a diferentes resoluções que forma uma estrutura hierárquica, sendo a base o nível 1, os filhos o nível 2, e assim sucessivamente. Cada célula pertencente a um nível superior é dividida para formar células de menor dimensão no nível inferior seguinte. Assim, sabe-se que o nível mais baixo apresenta uma maior resolução. Isto permite que os \textit{clusters} sejam encontrados recorrendo a uma pesquisa de cima para baixo (\textit{clustering} por divisão, como explicado na secção~\ref{subsec:hierar}), passando por cada nível até atingir o mais baixo, retornando no fim as células mais relevantes para a consulta especificada. A informação estatística de cada célula é calculada e armazenada para o processamento de consultas futuras. Também é necessário ter em atenção que apenas é considerado para este algoritmo um espaço bidimensional. Um outro algoritmo com características semelhantes é o CLIQUE~\citet{Agrawal1998}, que "identifica \textit{clusters} densos em sub-espaços de máxima dimensão", isto é, são detetados todos os \textit{clusters} em todos sub-espaços existentes e em que um ponto pode pertencer a vários \textit{clusters} em sub-espaços diferentes.

% Notas: Talvez falar mais do algoritmo CLIQUE

%\subsection{Dados de entrada para clustering} \label{subsec:inputdata}
%
%Os algoritmos de \textit{clustering} geralmente utilizam métricas específicas para o cálculo de similaridade ou distâncias entre objetos. 

\subsection{Funções de Distância} \label{subsec:dist}

As funções de distância ou similaridade têm um papel fulcral em todos os algoritmos de \textit{Clustering}. Existem inúmeras funções de distância usadas para diferentes tipos de atributos (ou variáveis)~\citet{Liu2011}. Em seguida serão apresentadas diferentes funções distância para diferentes atributos: numéricos, binários e nominal. Também serão apresentadas funções distância utilizadas para as dimensões temporal e de conteúdo. 

\subsubsection{Atributos Numéricos} \label{subsubsec: attrnum}

As funções de distância mais utilizadas para variáveis numéricas são a \textbf{Distância Euclidiana} e \textbf{Distância Manhattan}. É utilizado $ dist(x_{i}, x_{j}) $ para representar a distância entre duas instância de \textit{r} dimensões. Ambas as funções referidas anteriormente são casos especiais da função mais geral chamada \textbf{Distância Minkowski}~\citet{Liu2011}:

\begin{equation}
dist(x_{i}, x_{j}) = (|x_{i1} - x_{j1}|^h + |x_{i2} - x_{j2}|^h +...+ |x_{ir} - x_{jr}|^h)^\frac{1}{h},
\label{eq:mink}
\end{equation}
onde \textit{h} é um inteiro positivo.

Se \textit{h}=2, temos a \textbf{Distância Euclidiana},
\begin{equation} 
dist(x_{i}, x_{j}) = \sqrt{(x_{i1} - x_{j1})^2 + (x_{i2} - x_{j2})^2 +...+ (x_{ir} - x_{jr})^2}.
\label{eq: euclid}
\end{equation}

Se \textit{h}=1, temos a \textbf{Distância City-block (Manhattan)},
\begin{equation}
dist(x_{i}, x_{j}) = |x_{i1} - x_{j1}| + |x_{i2} - x_{j2}| +...+ |x_{ir} - x_{jr}|.
\label{eq: manhattan}
\end{equation}

Também, não menos importantes, são outras funções distância apresentadas em seguida:

\begin{description}
\item[Distância Euclidiana Ponderada]: A ponderação é atribuída através de pesos dados pela importância que cada atributo representa relativamente a outros atributos.

\begin{equation}
dist(x_{i}, x_{j}) = \sqrt{w_{1}(x_{i1} - x_{j1})^2 + w_{2}(x_{i2} - x_{j2})^2 +...+ w_{r}(x_{ir} - x_{jr})^2}.
\end{equation} 

\item[Distância Euclidiana Quadrática]: Trata-se de uma alteração da função \textbf{Distância Euclidiana}, elevando a mesma ao quadrado, o que faz com que seja progressivamente atribuído peso maior a pontos dos dados que estejam mais afastados.

\begin{equation}
dist(x_{i}, x_{j}) = (x_{i1} - x_{j1})^2 + (x_{i2} - x_{j2})^2 +...+ (x_{ir} - x_{jr})^2.
\end{equation}

\item[Distância Chebyshev]: Utilizada para casos em que há necessidade de definir dois pontos dos dados como diferentes, caso sejam diferentes em qualquer dimensão.

\begin{equation}
dist(x_{i}, x_{j}) = max(|x_{i1} - x_{j1}| + |x_{i2} - x_{j2}| +...+ |x_{ir} - x_{jr}|).
\label{eq:cheby}
\end{equation}

\end{description}

\subsubsection{Atributos Binários e Nominais}

As funções apresentadas anteriormente apenas podem ser utilizadas com atributos do tipo numérico, assim serão necessárias funções de distância especificas para atributos do tipo binário e nominal.

Uma variável binária é aquela que apenas pode assumir dois estados ou valores, sendo normalmente representado pelo valor 0 e 1. Mas estes estados não apresentam um ordem definida. Por exemplo, no caso de uma lâmpada, esta pode assumir apenas dois estados, ligado ou desligado, ou o género de uma pessoa, masculino ou feminino. Estes exemplos apresentam dois valores diferentes mas que não possuem qualquer ordem. As funções distância existentes para atributos binários são baseadas na proporção, sendo que a melhor maneira de representar é através de uma matriz confusão~\citet{Liu2011}.

\begin{figure}[h]
\centering
\includegraphics[width=0.8\linewidth]{./figures/matriz_confusao}
\caption{Matriz confusão de dois objetos com atributos binários}
\label{fig:matriz_confusão}
\end{figure}

Os atributos binários ainda podem ser divididos em dois tipos de atributos diferentes, os simétricos e os assimétricos, sendo em seguido apresentado as funções distância para ambos oa casos~\citet{Liu2011}.

\begin{description}
\item[Atributos simétricos: ] Um atributo é simétrico quando ambos os estados (0 ou 1) têm a mesma importância e o mesmo peso, tal como ocorre no exemplo dado anterior com o atributo género (masculino e feminino). Para este caso, a função distância mais utilizada é designada por \textit{simple matching distance}, que corresponde à proporção de incompatibilidade ou desacordo(equação~\ref{eq:symetric}).  

\begin{equation}
 dist(x_{i}, x_{j}) =  \frac{b + c}{a + b + c + d}
 \label{eq:symetric}   
\end{equation}

\item[Atributos assimétricos: ] Um atributo é assimétrico se um dos estados apresenta maior importância ou valor do que o outro. Normalmente o estado mais valioso é o que ocorre com menor frequência. No nosso caso iremos considerar o estado 1 como o mais valioso. Assim, a função distância mais frequentemente utilizada para atributos assimétricos é a \textit{Jaccard distance}:

\begin{equation}
 dist(x_{i}, x_{j}) =  \frac{b + c}{a + b + c}
\label{eq:asymetric}   
\end{equation}

\end{description}

No caso de atributos nominais com mais de dois estados ou valores, a função distância mais utilizada, é baseada na \textit{simple matching distance}. Dados dois objetos \textit{i} e \textit{j}, \textit{r} corresponde ao número total de atributos e o \textit{q} ao número de valores que são mutuamente correspondidos entre os objetos \textit{i} e \textit{j}:

\begin{equation}
 dist(x_{i}, x_{j}) =  \frac{r + q}{r}
\end{equation}

\subsubsection{Dimensão Temporal} \label{subsubsec: time}

O tempo é representado apenas por uma dimensão, sendo que para calcular a distância, por exemplo, entre dois tweets $ t_{i} $ e $ t_{j} $, apenas é necessário calcular a diferença dos tempos entre os mesmos. Supondo que os valores dos tempos são respetivamente $ \Delta_{i} $ e $ \Delta_{j} $, o intervalo de tempo pode ser definido pela seguinte equação:

\begin{equation}
dist^{T}( t_{i}, t_{j}) = |\Delta_{i} - \Delta_{j}|  
\end{equation}

%Sendo que, para a dimensão temporal, também é possível utilizar a função distância euclidiana (equação~\ref{eq: euclid}).

\subsubsection{Dimensão Espacial} \label{subsubsec: space}

Ao contrário da dimensão temporal, a dimensão espacial apresenta mais do que uma dimensão, latitude e longitude. Estas apresentam-se sobre a forma numérica, sendo possível o cálculo da distância entre dois objetos através de funções de distância para atributos numéricos como referido anteriormente (\ref{subsubsec: attrnum}). Assim, para o cálculo entre pontos distribuídos num espaço poder-se-á recorrer à função Minkowski (equação~\ref{eq:mink}), à função Euclidiana (equação\ref{eq: euclid}), à função Manhattan (equação~\ref{eq: manhattan}) ou mesmo à função de Chebychev (equação~\ref{eq:cheby}), sendo que no caso mais específico de uma distribuição espacial geográfica, em que os pontos possuem latitude e longitude, é considerada mais apropriada a utilização da função distância Haversine pois esta toma em consideração a forma esférica da Terra~\cite{Montavont2006}. Assim, obtendo um par de objetos $ x_{i} $ e $ x_{j} $ distanciados geograficamente, são consideradas a latitude $ \phi_{x_{i}} $ e $ \phi_{x_{j}} $ e a longitude $ \lambda_{x_{i}} $ e $ \lambda_{x_{j}} $ para determinar a distância entre os objetos através da equação~\ref{eq:hav}

\begin{equation}
dist^{Sp}( x_{i}, x_{j}) = 2R\sin^{-1}\left( \left[ \sin^{2}(\frac{\phi_{x_{i}}-\phi_{x_{j}}}{2})+\cos\phi_{x_{i}}\cos\phi_{x_{j}}\sin^{2}(\frac{\lambda_{x_{i}}-\lambda_{x_{j}}}{2})\right] ^{0.5}\right) 
\label{eq:hav} 
\end{equation}

onde $ R $ representa o raio da Terra e que determina as unidades do resultado retornado pela função, sendo comum a utilização das unidades no sistema internacional (SI), o metro, podendo também ser representado em quilómetros devido ao fator de escala. 


% % % NOVA SECÇÃO % % %

\section{TweeProfiles} \label{sec:tweep}

Esta dissertação pretende dar continuidade e um trabalho designado por TweeProfiles~\cite{Cunha2013}. O TweetProfiles é uma ferramenta de análise de dados recolhidos no Twitter e visualização \textit{clusters} em várias dimensões, tais como, espacial, temporal, de conteúdo (o texto dos tweets) e social. Esta secção faz um breve introdução e descrição sobre esta ferramenta.

\subsection{Descrição e objetivos}

O TweeProfiles aborda o problema de identificar perfis de tweets envolvendo múltiplos tipos de informação: espacial, temporal, social e de conteúdo. A informação espacial, trata-se da informação de localização das mensagens de texto partilhadas no Twitter, a temporal é relativa à data de publicação do tweet, a social às ligações entre os utilizadores, e por fim o conteúdo que neste caso é relativo ao texto contido em cada tweet. 

Os objetivos do TweeProfiles foi o desenvolvimento de uma metodologia de Data Mining que identificasse perfis de tweets que combinem de forma flexíveis as várias dimensões consideradas, a criação de uma ferramenta de visualização para representar os resultados obtidos e a sua aplicação a um caso de estudo na twittosfera portuguesa.
 
A ferramenta de visualização está desenhada para uma utilização dinâmica e intuitiva, direcionada para a representação dos perfis de uma forma compreensível e interativa. Esta apresenta vários widgets capazes de representar os padrões obtidos.
O caso de estudo que o TweeProfiles aborda dados georeferenciados do Socialbus (antes designado como TwitterEcho). No entanto, esta ferramenta é adequada para tratar quaisquer mensagens georeferenciadas provenientes do Twitter.

\subsection{Resultados ilustrativos} \label{subsec:tpresult}

Através do TweeProfiles podemos visualizar \textit{clusters} nas suas diversas dimensões. Esta ferramenta apresenta três secções distintas para visualização e incluí ainda controlos para navegação e seleção de determinados parâmetros. Uma das secções é constituída por um mapa onde é possível visualizar geograficamente os \textit{clusters}, como podemos ver na Figura~\ref{fig:maptweep1}, em que são representados por círculo. Outra das secções é um gráfico que apresenta a distribuição temporal dos \textit{clusters}, onde é possível visualizar a data de início de fim de um determinado \textit{cluster}, como representado na Figura~\ref{fig:temptweep1}. 
\begin{figure}[h]
\centering
\includegraphics[width=0.9\linewidth]{./figures/tweeprofiles/maptweep1}
\caption{Exemplo ilustrativo da distribuição espacial dos \textit{clusters} calculada apenas com a consideração da dimensão espacial. \textit{Retirada de}~\cite{Cunha2013}}
\label{fig:maptweep1}
\end{figure}
Por fim, é nos apresentada uma secção onde é possível visualizar informação mais especifica sobre os \textit{clusters}, onde é incluído a informação relativa às dimensões de conteúdo e social com a representação das palavras mais utilizadas e as ligações entre utilizadores respetivamente nesse \textit{clusters}. No caso dos controlos, estes permitem selecionar um dos três intervalos de tempo existentes e o os pesos para cada uma das dimensões, que podem assumir valores entre 0\% e 100\% com incrementos de 25\%. Cada dimensão pode assumir como peso assim um dos seguintes valores percentuais: \{0, 25, 50, 75, 100\}, sendo que a soma dos pesos das diferentes dimensões deve ser igual a 1. 

Em seguida são apresentados alguns resultados ilustrativos obtidos através da ferramenta TweeProfiles, e que demonstram o seu funcionamento global. Este resultados serão baseados apenas nos apresentados por Tiago Cunha~\cite{Cunha2013}, sendo ilustrados resultados de \textit{clusters} em Portugal para diferentes combinações possíveis. 

\begin{figure}[h]
\centering
\includegraphics[width=0.9\linewidth]{./figures/tweeprofiles/tempexample1}
\caption{Exemplo ilustrativo da distribuição temporal de um \textit{cluster}. \textit{Retirada de}~\cite{Cunha2013}}
\label{fig:temptweep1}
\end{figure}

Na Figura~\ref{fig:tweepex1} são apresentadas quatro combinações entre a dimensão do conteúdo e as restantes, exceto no primeiro caso~\ref{fig:sfig1} onde podemos ver um \textit{cluster} em Portugal apenas tendo em consideração a dimensão do conteúdo. Já no caso da Figura~\ref{fig:sfig2} o peso é distribuído da mesma forma entre o conteúdo e a dimensão espacial, resultando em dois \textit{clusters}. Na Figura~\ref{fig:sfig3}, tal como no caso anterior, é distribuído o peso de igual forma entre o conteúdo, e neste caso a dimensão temporal. O resultado obtidos no caso de Portugal, assemelha-se ao do primeiro caso da Figura~\ref{fig:sfig1}. Por fim, temos o caso da Figura~\ref{fig:sfig4} onde o peso é de 50\% para o conteúdo e 50\% para a dimensão social, tendo resultado para o caso de Portugal, dois \textit{cluster}. 

\begin{figure}[h]
\centering
\begin{subfigure}[b]{.2\textwidth}
  \centering
  \includegraphics[width=0.98\linewidth]{./figures/tweeprofiles/extp1}
  \caption{\textit{Clusters} em Portugal: Conteúdo 100\%}
  \label{fig:sfig1}
\end{subfigure}%
\quad
\begin{subfigure}[b]{.2\textwidth}
  \centering
  \includegraphics[width=0.99\linewidth]{./figures/tweeprofiles/extp2}
  \caption{\textit{Clusters} em Portugal: Conteúdo 50\% + Espacial 50\%}
  \label{fig:sfig2}
\end{subfigure}
\quad
\begin{subfigure}[b]{.2\textwidth}
  \centering
  \includegraphics[width=1\linewidth]{./figures/tweeprofiles/extp3}
  \caption{\textit{Clusters} em Portugal: Conteúdo 50\% + Temporal 50\%}
  \label{fig:sfig3}
\end{subfigure}
\quad
\begin{subfigure}[b]{.2\textwidth}
  \centering
  \includegraphics[width=0.99\linewidth]{./figures/tweeprofiles/extp4}
  \caption{\textit{Clusters} em Portugal: Conteúdo 50\% + Social 50\%}
  \label{fig:sfig4}
\end{subfigure}
\caption{Exemplos de resultados de \textit{clusters} com pesos diferentes para as diferentes dimensões.~\textit{Retiradas de}~\cite{Cunha2013}}
\label{fig:tweepex1}
\end{figure}

Outra das combinações apresentadas, foi a distribuição igual por todas as dimensões. Este caso é apresentado na Figura~\ref{fig:tweepex2} onde todas as dimensões possuem o peso de 25\% cada. Neste caso foram obtidos três \textit{cluster} em Portugal.

\begin{figure}[h]
\centering
\includegraphics[width=0.35\linewidth]{./figures/tweeprofiles/extp5.png}
\caption{\textit{Clusters} em Portugal: Conteúdo 25\% + Espacial 25\% + Temporal 25\% + Social 25\%.~\textit{Retirada de}~\cite{Cunha2013}}
\label{fig:tweepex2}
\end{figure}

Por fim, na Figura~\ref{fig:tweepex3} é possível visualizar a secção que apresenta ao utilizador informação sobre um \textit{cluster} selecionado, o conteúdo desse \textit{cluster}, como por exemplos, as palavras mais relevantes, e o grafo com as ligações sociais. É possível ainda visualizar com mais detalhe de um determinado tweet, como o nome do utilizador, informação temporal e espacial.

\begin{figure}[h]
\centering
\includegraphics[width=1.0\linewidth]{./figures/tweeprofiles/extp6.png}
\caption{Visualização de informação mais detalhada de um \textit{cluster} incluindo o seu grafo social, e da informação relativa a um determinado tweet.~\textit{Retirada de}~\cite{Cunha2013}}
\label{fig:tweepex3}
\end{figure}

\subsection{Prós e contras}

Como vimos na Secção~\ref{subsec:tpresult}, a ferramenta TweeProfiles permite uma grande variedade de combinações que produzem resultados que podem variar no número e posição geográfica dos \textit{clusters} consoante a influência e peso e das dimensões em consideração. 

Este projeto de dissertação não tem como objetivo melhorar O TweeProfiles, mas sim apresentar uma abordagem alternativa de visualização de análise de dados em redes sociais utilizando conteúdo diferente. Por este motivo, nesta secção somente abordamos os prós e contras relativamente à abordagem das dimensões utilizadas.

Uma das vantagens da utilização do texto como dimensão de conteúdo, é o facto de no Twitter este se apresentar como a maior fonte de informação, pois os tweets na sua maioria apresentam texto. Mesmo adicionando a dimensão espacial, que implica a existência de uma referência geográfica contida na informação dos tweets, o número de tweets disponíveis continua a ser relativamente razoável para uma análise dos dados, tal como podemos confirmar em~\cite{Cunha2013}. Apesar disso, é possível verificar qu nos resultados apresentados existe muito conteúdo em forma de texto que apresenta informação muitas vezes com pouca relevância, como é o caso da partilha da sua localização através de outros serviços onde é utilizado um texto pré-definido pelo serviço utilizado, indicando apenas o local onde se encontra o utilizador, como podemos ver pela Figura~\ref{fig:tweepex4}.

\begin{figure}[h]
\centering
\includegraphics[width=1.0\linewidth]{./figures/tweeprofiles/extp7.png}
\caption{Tweets pertencentes a um \textit{cluster} para a seguinte distribuição de pesos: Conteúdo 25\% + Espacial 25\% + Temporal 25\% + Social 25\%.~\textit{Retirada de}~\cite{Cunha2013}}
\label{fig:tweepex4}
\end{figure}

O TweeProfiles revela ainda que a dimensão espacial, quando tida em consideração, tem um influência visivelmente interessante nos resultados quando apresentados no mapa. Já a dimensão temporal, não revelou apresentar uma grande influência como vimos no caso apresentado na Figura~\ref{fig:sfig3}, mas esta permite um controlo e visualização de informação interessante através do gráfico de tempo como representado na Figura~\ref{fig:temptweep1}. No caso da dimensão social, verifica-se que apresenta alguma influência na divisão dos \textit{clusters} mas a visualização da sua informação através do grafo, apresenta-se pobre e de difícil compreensão para o utilizador.

Assim, concluí-se que a utilização das dimensões espaço-temporal fazem todo o sentido no desenvolvimento desta dissertação, sendo necessária a sua inclusão para a extensão do TweeProfiles. Já no caso do conteúdo de texto, este não será tido em conta de forma a ser analisado se existe uma mais valia em realizar uma análise em tweets através exclusivamente do conteúdo visual partilhado pelos utilizadores com as possíveis combinações com as dimensões temporal e espacial. A dimensão social, também não será incluída por esta não adicionar ainda não apresentar a informação de forma clara para um utilizador, sendo necessário um estudo mais aprofundado sobre esta temática, que não faz parte dos objetivos desta dissertação.

% % % NOVA SECÇÃO % % %

\section{Representação de Informação Visual} \label{sec:represent}

Na secção~\ref{sec:cluster} foi apresentado o conceito e características da tarefa de \textit{clustering} de uma forma geral. Nesta secção serão expostas formas de representar imagens como dados. 

Como o objetivo desta dissertação passa por a realização da tarefa de \textit{clusterng}, utilizando como dados as imagens partilhadas no serviço Twitter, é necessário utilizar formas eficientes para descrever cada imagem de uma forma compacta e de forma a que seja descrito o conteúdo geral das imagens. É importante salientar que a tarefa de \textit{clustering} em imagem é muitas vezes associada à técnica de segmentação de imagem~\citet{Forsyth2011}, em que se pretende distinguir objetos individuais, não sendo este o objetivo deste projeto de dissertação. Pretende-se sim que para tarefa de \textit{clustering} o objeto representativo de uma imagem descreva esta pelo seu conjunto, tal como referido anteriormente. 

\subsection{Representação Matricial} \label{subsec:matrix}

Uma imagem pode ser vista como um objeto (ou instância), sendo computacionalmente representada como uma matriz (um vetor bi-dimensional) de pixels. A matriz de pixels descreve assim a imagem como N x M \textit{m}-bit pixels, onde N corresponde ao número de pontos ao longo do eixo horizontal, M o número de pontos ao longo do eixo vertical e \textit{m} o número de bits por pixel que controla os níveis de brilho. Com \textit{m} bits temos uma gama de valores para o brilho de $ 2^m $, que varia entre 0 e $ 2^m - 1 $. Assim se o valor de \textit{m} for 8, os valores de brilho de cada pixel de uma imagem podem variar entre 0 e 255, que normalmente correspondem ao preto e branco respetivamente, sendo que os valores intermédios correspondem ao tons de cinza~\citet{Nixon2002}.

No caso de imagens a cores, o principio é idêntico, no entanto ao invés de se usar apenas um plano, as imagens a cores são representadas por 3 componentes de intensidade, designado por modelo \textit{RGB}, a que corresponde respetivamente às cores vermelho (Red), verde (Green) e azul (Blue). Para além deste esquema de cores, também existe outros como o CMYK composto pelas componentes de cor, azul turquesa, magenta, amarelo e preto. Usando qualquer esquema de cores, existem 2 métodos principais para representar a cor do pixel. No primeiro método é utilizado um valor inteiro para cada pixel, sendo esse valor como um índice para uma tabela, também conhecida como palete da imagem, com a correspondência à intensidade de cada componente de cor. Este método tem como vantagem o facto de ser eficiente na utilização da memória, pois apenas é guardado um plano da imagem (os índices) e a palete (tabela). Por outro lado, tem como desvantagem o facto de normalmente ser usado um conjunto reduzido de cores o que provoca uma redução da qualidade da imagem. Já o segundo método consiste na utilização de vários planos da imagem para armazenar a componente de cor de cada pixel. Este representa a imagem com mais precisão pois considera muito mais cores. O formato mais usual é 8 bits para cada uma das 3 componentes, no caso do RGB. Assim, são utilizados 24 bits para representar a cor de cada pixel, o que permite que uma imagem possa conter mais de 16 milhões de cores simultaneamente. Como era de esperar, isto envolve um custo grande na utilização de memória, que mesmo apresentando-se como uma desvantagem, com a constante redução do custo das memórias esta passou ser uma boa alternativa à apresentada anteriormente~\citet{Nixon2002}.

%ver melhor esta conclusão
Em suma, a representação matricial é uma forma fácil de representar uma imagem, mas ao contrário dos objetivos desta dissertação, não consegue fazê-lo de uma forma compacta, sendo necessária a existência de grandes quantidades de memória devido à sua dimensionalidade. Assim, esta não se apresenta como uma boa solução para o a extração da informação das imagens.

\subsection{Histogramas} \label{subsec:hist}

Outras das formas de representar a informação de uma imagem é através de um histograma. Um histograma de uma imagem apresenta a frequência de ocorrência de níveis individuais de brilho, representado através um gráfico que mostra o número de pixels da imagens com um determinado nível de brilho. No caso de pixels representados por 8-bit, o brilho vai variar de 0 (preto) até 255 (branco)~\citet{Nixon2002}. Também pode ser apresentado informação de cor sobre uma imagem através de um histograma, sendo para isso necessário apresentar 3 histogramas diferenciados, um para cada componente de cor, no caso do esquema RGB.
A figura~\ref{fig:lenahist} apresenta um exemplo de um histograma de uma imagem com tons cinza, onde são representados o número de pixeis para cada nível diferente de cinzento.

\begin{figure}[h]
\centering
\includegraphics[width=0.8\linewidth]{./figures/histlena}
\caption{Imagem em tons cinza e respetivo histograma }
\label{fig:lenahist}
\end{figure}

\subsection{Descritores de Cor}

A cor apresenta-se como um importante atributo da imagem para o olho humano e processamento por computador. Nesta secção são apresentados vários descritores de cor, utilizados para extração de informação e reconhecimento de similaridade em imagens. Por exemplo, o histograma de cores, referido na secção~\ref{subsec:hist}, é um dos descritores de cor mais utilizados para caracterizar a distribuição da cor de uma imagem, mas apresenta uma baixa eficiência. Assim, em seguida é apresentado descritores de cor considerados pelo MPEG-7~\cite{Manjunath2001, Christopoulos2000, Cieplinski2001, Ite-vil}, que apresentam eficiência superior aos histogramas de cor.


\subsubsection{Espaços de Cor} \label{subsubsec:space}

Nesta secção é apresentado os vários descritores de espaços de cor especificado no MPEG-7~\cite{Ite-vil}. Existe uma vasta seleção de espaços de cores, tais como, RGB, YCbCr, HSV, HMMS, Monocromático e Matriz linear de transformação com referência a RGB. Estes são usados por outro descritor de cor, mais especificamente, o descritor de dominância de cor que será falado posteriormente. É utilizado também, um sinalizador para indicar a referência a uma cor primária e de mapeamento de um valor de referência do branco padrão. 

Em espaços de cor, as componentes de cor são definidas como entidades de valor continuo, sendo que podem ser representadas por valores discretos através de uma quantização uniforme, em que é especificado um número de níveis de quantização para cada componente de cor no espaço de cor. A única exceção é o espaço de cor HMMD.

O espaço de cor RGB é um dos modelos referidos mais utilizados, que apresenta três componentes distintas, vermelho, verde e azul, tal como foi referido no secção~\ref{subsec:matrix}. Neste modelo é utilizado a combinação das 3 cores primárias para representar as diferentes cores. O modelo YCbCr provém do padrão MPEG-1/2/4~\cite{Ite-vil} e é definido pela transformação linear do espaço de cor RGB como demonstrado na equação~\ref{eq:ycbcr}:

\begin{eqnarray}
&& Y = 0.299\times R + 0.587\times G + 0.114\times B\nonumber\\
&& Cb = -0.169\times R - 0.331\times G + 0.500\times B \nonumber\\
&& Cr = 0.500\times R - 0.419\times G - 0.081\times B \label{eq:ycbcr}
\end{eqnarray} 

Para o espaço de cor Monocromático, é usado apenas a componente Y do modelo YCbCr. 

O espaço de cor HSV apresenta uma especificação mais complexa, tendo sido desenvolvido para fornecer uma representação mais intuitiva e para se aproximar mais do sistema visual humano. A transformação do modelo RGB para o HSV não é linear, mas é reversível~\cite{Manjunath2001}. Uma das componentes é a matiz (H - \textit{Hue}), que representa a componente de cor espectral dominante na sua forma mais pura, como o verde, amarelo, azul e vermelho. Ao ser adicionado branco à cor, esta sofre uma alteração, sendo que, adicionando mais branco, menos saturada se torna a cor. A saturação (S - \textit{Saturation}) é precisamente outra das componentes deste modelo. Por fim, o valor (V - \textit{Value}) corresponde ao brilho de cor.

O espaço de cor HMMD (\textit{Hue-Max-Min-Diff})~\cite{Manjunath2001, Ite-vil} é mais recente, que é caracterizado pela componente matiz, tal como o modelo HSV, pelo \textit{max} e \textit{min}, que são respetivamente o máximo e mínimo entre os valores R, G e B. Para descrever este modelo, também é utilizado a componente \textit{Diff}, que corresponde à diferença entre o \textit{max} e \textit{min}. Para representar este espaço de cor, apenas é necessário três dos quatro componentes referidos anteriormente, como por exemplo, {\textit{Hue, Max, Min}} ou {\textit{Hue, Diff, Sum}}, onde \textit{Sum} pode ser definida pela equação~\ref{eq:sum}.

\begin{equation}
Sum = \frac{Max + Min}{2}
\label{eq:sum}
\end{equation}

\subsubsection{Cor Dominante}

O descritor de cor dominante fornece uma representação compacta das cores de uma imagem ou da região da imagem. Este apresenta a distribuição das cores mais representativas na imagem. Ao contrário do descritor de cor por histograma, na especificação do descritor de cor dominante, as cores mais representativas são calculadas a partir de cada imagem, em vez de ser fixado no espaço de cor, permitindo assim, uma representação das cores mais exata e compacta, presentes numa região de interesse.

O descritor de cor dominante pode ser definido como,
\[ F = {{c_{i}, p_{i}, v_{i}},s}, (i=1,2,...,N) \]
onde N é o número de cores dominantes. Cada valor $ c_{i} $ da cor dominante é um vetor de valores das componentes do espaço de cor correspondente (por exemplo, um vetor de 3 dimensões no espaço de cor RGB). O valor $ p_{i} $ é a fração de pixels na imagem ou região da imagem (normalizado para um valor entre 0 e 1) que corresponde à cor $ c_{i} $, sendo $  \sum_i p_{i} = 1 $. O opcional $ v_{i} $ descreve a variação dos valores de cor dos pixels em um \textit{cluster} em torno da cor representativa correspondente. Por fim, a coerência espacial $ s $ é um único número  representa a homogeneidade espacial global das cores predominantes na imagem~\cite{Ite-vil}.

\subsubsection{Cor Escalável}

O descritor de cor escalável, pode ser interpretado como um esquema de codificação base que recorre à transformada de Haar, que é aplicada aos valores do histograma de cor no espaço de cor HSV (referido na secção~\ref{subsubsec:space}). De uma forma mais específica, o descritor de cor escalável, extrai, normaliza e mapeia de forma não linear os valores do histograma, numa representação inteira a 4-bit, dando assim mais relevância a valores mais pequenos. A transformada de Haar, é assim aplicada aos valores inteiros a 4-bit através das barras do histograma. 

A extração do descritor é realizada com computação de um histograma de cor com 256 níveis no espaço de cor de HSV com a componente matiz (H) quantificada a 16 níveis, e a saturação (S) e o valor (V) quantificado cada um para 4 níveis~\cite{Christopoulos2000}. 

A aplicação típica do descritor é na a busca de similaridade numa base de dados com conteúdo multimédia e pesquisa em enormes base de dados. 

%\subsubsection{Grupo de \textit{Frames} / Grupo de Imagens}
%
%O descritor de cor Grupo de Frames / Grupo de Imagens (GoF/GOP) é sobretudo utilizado para a representação conjunta de cores para várias imagens ou várias \textit{frames} de um segmento de vídeo, contíguas ou não contíguas. Este baseia-se em histogramas, que capturam de forma confiável o conteúdo da cor de várias imagens ou \textit{frames} de vídeo. Normalmente para um grupo de \textit{frames} ou imagens, é selecionado uma frame chave ou imagem chave, que representará as características relacionadas com o grupo. Os métodos são altamente dependentes da qualidade da seleção da amostra representativa, o que pode levar a resultados pouco fiáveis caso não seja bem executada~\cite{Ite-vil}.
%
%A estrutura do descritor GoF / GoP é idêntica à do cor escalável, com a exceção do campo agregação, que especifica como os pixels da cor de diferentes imagens/\textit{frames} foram combinadas antes da extração do histograma de cor. Os valores possíveis são média, mediana e cruzamento. 
%
%Uma das aplicações deste descritor, é a pesquisa em conjuntos grandes de imagens, para encontrar \textit{clusters} de imagens semelhantes através da cor, em que é utilizado a interseção de histogramas como medida de similaridade da cor. A interseção de histogramas é obtido calculando o valor mínimo de cada barra de cor do histograma ao longo das \textit{frames}/imagens e atribui esse valor às barras de cor do histograma resultante. A interseção encontra as cores mínimas comuns nas \textit{frames}/imagens, e portanto, pode ser utilizado em aplicações que requerem a deteção de um elevado grau de correlação da cor~\cite{Christopoulos2000}. 

\subsubsection{Estrutura de Cor}

Este descritor é uma generalização do histograma de cores, que apresenta algumas características espaciais da distribuição de cores numa imagem. Este tem a particularidade de, para além de apresentar o conteúdo da cor de forma semelhante a um histograma de cor, também apresentar informações sobre a estrutura de uma imagem, sendo esta a característica diferenciadora deste descritor de cor. Em vez de considerar cada pixel separadamente, o descritor recorre a uma estrutura de 8x8 pexels que desliza sobre a imagem. Ao contrário do histograma de cor, este descritor consegue distinguir duas imagens em que uma determinada cor está presente em quantidades iguais, mas que apresenta uma estrutura num dos grupos de pixels 8x8 com uma cor diferente nas duas imagens. Os valores de cores são representadas no espaço de cor HMMD com cone duplo, sendo o espaço quantificado de maneira não uniforme em 32, 64, 128 ou 256 níveis. 
Cada valor de amplitude de um nível é representado por um código de 8 bits. Este descritor apresenta um bom desempenho na tarefa de recuperação de imagens baseado na similaridade~\cite{Modi2008}.

%\subsubsection{Disposição de Cor}
%
%O descritor de disposição de cor, caracteriza a distribuição espacial de cor de uma imagem. Este usa um vetor de cores representativas de uma imagem, expressas no espaço de cor YCbCr. O tamanho do vetor é fixado em 8x8 elementos para garantir invariância da escala do descritor. As cores representativas da imagem podem ser selecionadas de diversas maneiras, mas a mais simples é através do cálculo da média de cor do bloco de imagem correspondente. O descritor disposição de cor, pode ser usado para em pesquisa rápida de bases de dados de imagens~\cite{Cieplinski2001}.

\subsection{Descritores de Textura}

A textura das imagens é uma característica visual importante, que tem muitas aplicações na recuperação, navegação e indexação de imagens. Existem três descritores de textura, referenciados na norma MPEG-7~\cite{Wu2001}. Em seguida é apresentada uma pequena descrição dos mesmos.

\subsubsection{Descritor de Textura Homogénea}

O descritor de textura homogénea (HTD - \textit{Homogeneous Texture Descriptor}) descreve a distribuição estatística da textura de uma imagem. Existem neste descritor 62 interfaces de recurso, sendo 2 no domínio espacial e 60 no domínio das frequências. No domínio espacial, é extraído a média e o desvio padrão de uma imagem. No domínio das frequências, o espaço é dividido em 30 canais, sendo calculado o valor energético e o valor do desvio de energia da resposta do filtro Gabor em cada canal~\cite{Wu2001, Shao2009}. Este desenho baseia-se no facto de a resposta do córtex visual possuir banda limitada e do facto do cérebro decompor o espectro em bandas na frequência espacial~\cite{Wu2001}.
O descritor de textura homogénea é essencialmente utilizado em aplicações de recuperação de imagens por similaridade. 

\subsubsection{Descritor de Histograma de Borda}

O descritor de histograma de borda (EHD - \textit{Edge Histogram Descriptor}) apresenta-se sobre a forma de um histograma de 80 níveis, que representa a distribuição de borda local de uma imagem. Este descreve as bordas em cada sub-imagem. Estas sub-imagens são obtidas através da divisão da imagem numa grelha 4x4 como pode ser visto na Figura~\ref{fig:ehd}. Existem 5 tipos de classificação diferentes das bordas de cada sub-imagem, sendo elas: vertical, horizontal, 45-graus, 135-graus e não direcional~\cite{Wu2001}. 

Este descritor é utilizado na recuperação de imagens, como por exemplo, imagens naturais ou de esboço, devido à sua textura homogénea. É também suportado por este descritor, a pesquisa baseada em blocos de imagem.  

\begin{figure}
\centering
\includegraphics[width=0.7\linewidth]{./figures/ehd}
\caption{Sub-imagem e bloco de imagem. \textit{Retirada de}~\cite{Wu2001}}
\label{fig:ehd}
\end{figure}

\subsubsection{Descritor de Navegação Percetual}

O descritor de navegação percetual (HTD - \textit{Perceptual Browsing Descriptor}) foi projetado para navegação em base de dados, mas principalmente para quando essa navegação necessita de recursos com sentido percetual~\cite{Wu2001}. Este descritor é bastante compacto, que requer apenas 12 bits (máximo) para caracterizar regularidade (2 bits), direcionamento (3 bits x 2) e grosseirismo (2 bits x 2) da textura de uma imagem. A regularidade de uma textura pode apresentar valores numa escala entre 0 e 3, em que 0 indica uma textura irregular ou aleatória e 3 indica um padrão com direção e grosseirismo bem definidos. O direcionamento de uma textura é quantizada em 6 valores, variando de 0 a 150 em degraus de 30. Por fim, o grosseirismo de uma textura está relacionado com a escala e resolução de uma imagem. É quantizado em 4 níveis de 0 a 3, sendo 0 para um grão fino e 3 para uma textura grosseira. Estes valores têm uma relação com a divisão do espaço de frequência usado no cálculo do descritor de textura homogénea (HTD)~\cite{Manjunath2001}

\subsection{Descritores de Forma}

O descritores de forma são dos descritores mais poderosos no reconhecimento de objetos. Isto deve-se ao facto de os seres humanos serem exímios no reconhecimento de objetos característicos exclusivamente através da suas formas, provando que a forma muitas vezes possui informação semântica~\cite{Bober2001}.  

\subsubsection{Descritor de Forma Baseado em Região}
O descritor de forma baseado em região (RSD - \textit{Region-based Shape Descriptor})~\cite{Bober2001} apresenta a distribuição de um pixel dentro de uma região de um objeto em 2 dimensões. Este permite a descrição de objetos simples, com ou sem buracos (figura~\ref{fig:shape1}), mas também permite a descrição de objetos mais complexos, que contém múltiplas regiões sem ligação.

\begin{figure}[h]
\centering
\includegraphics[width=0.7\linewidth]{./figures/shape1}
\caption{Exemplo de formas de objetos que podem ser descritas eficazmente pelo descritor baseado em região. \textit{Retirada de}~\cite{Bober2001} .}
\label{fig:shape1}
\end{figure}

As principais características deste descritor são~\cite{Bober2001}:

\begin{itemize}
\item Fornece uma forma compacta e eficiente de descrever várias regiões disjuntas;
\item Quando, no processo de segmentação de um objeto, ocorrem sub-regiões sem ligação, o objeto ainda pode ser recuperado, desde que a informação de quais as regiões que foram divididas seja mantida e usada na extração do descritor;
\item Apresenta uma boa robustez à segmentação de ruído.
\end{itemize}

\subsubsection{Descritor de Forma Baseado no Contorno}

O descritor de forma baseado no contorno (CSD - \textit{Contour-based Shape Descriptor})~\cite{Bober2001} fundamenta-se na representação da curvatura espaço-escala (CSS - \textit{Curvature Scale-Space}) do contorno. O contorno é uma propriedade importante na identificação de objetos semanticamente semelhantes. É também bastante eficiente em aplicações onde a forma de objetos são muito variáveis, ou quando por exemplo, existem deformações de perspetiva. Esta apresenta um boa eficiência mesmo perante a existência de ruído nos contornos. 

As principais características deste descritor são~\cite{Bober2001}: 

\begin{itemize}
\item Consegue distinguir objetos que apresentem formas semelhantes mas que a forma do contorno apresenta propriedades bem diferenciadoras;
\item Tem a capacidade de encontrar formas que são semanticamente similar para os seres humanos, mesmo quando existe uma significativa variabilidade intra-classe;
\item É eficiente mesmo em casos de deformações não rígidas;
\item É eficiente mesmo em casos de distorções do contorno devido a variações de perspetiva, sendo uma situação muito comum em imagens e vídeo.
\end{itemize}
%(3D SD - 3-D Shape Descriptor)

\subsection{Descritores Locais} \label{subsec:desclocal}

Um descritor local permite a localização das estruturas locais de uma imagem de forma repetitiva. Estas são codificadas de modo a que sejam invariantes a transformações das imagens, tais como a translação, rotação, mudanças de escala ou deformações. Assim, estes descritores podem ser utilizados para representar uma imagem e podem ser utilizados para diversos fins, tais como, reconhecimento de objetos, reconhecimento de cenas, perseguição de movimento, correspondência entre imagens ou mesmo obtenção de estruturas 3D de múltiplas imagens. 
Para a extração das características deste descritor é necessário utilizar um processo com as seguintes etapas~\cite{Gauman2010}:

\begin{itemize}
\item Encontrar um conjunto de pontos chave;
\item Definir uma região em torno de cada ponto-chave numa escala invariante;
\item Extrair e caracterizar o conteúdo da região;
\item Calcular o descritor da região normalizada;
\item Combinar os descritores locais.
\end{itemize}

Em seguida são apresentados duas das técnicas mais representativos dos descritores locais, o SIFT e o SURF.

\subsubsection{SIFT} \label{subsubsec:sift}

O descritor SIFT (\textit{Scale-Invariant Feature Transform})~\cite{Lowe1999, Lowe2004} é um descritor local que transforma uma imagem numa grande coleção de vetores de características locais invariantes a translação, rotação, mudanças de escala, e parcialmente invariante a mudanças de iluminação. Pode assim ser utilizado para detetar correspondência entre imagens com diferentes visões de objetos ou cenas.
Este descritor apresenta como característica interessante o facto de compartilhar uma série de propriedades em comum com as respostas dos neurónios do lobo temporal na visão dos primatas. Os pontos-chave SIFT derivados de uma imagem são usados na indexação numa abordagem de vizinho mais próximo, para encontrar objetos candidatos.
% O modelo recorre a uma votação pela transformada Hough e a uma estimativa final pelo método dos mínimos quadrados. Quando 3 pontos chave estiverem em acordo, pode se afirmar que existe uma forte possibilidade da presença de um objeto.

Como foi indicado anteriormente, são necessários levar a cargo uma lista de etapas bem definidas para obtenção de descritores locais, sendo que as etapas para o descritor SIFT as seguintes:

\begin{enumerate}
\item \textbf{Deteção de extremos}:  A deteção de extremos (máximos e mínimos) é conseguida através da procura realizada em várias escalas e localizações, de modo a serem extraídos pontos de interesse invariáveis à escala e rotação, através da diferença de filtros gaussianos. Estes pontos de interesse ou pontos chave correspondem a estes extremos para várias escalas. 
Um filtro gaussiano passa baixo é dado pela convolução entre uma imagem \textit{I} e a função \textit{G}:

\begin{equation}
L\left ( x, y, \sigma  \right ) = G\left ( x, y, \sigma  \right ) * I\left ( x, y \right )
\end{equation}

onde ,
\begin{equation}
G\left ( x, y, \sigma  \right ) = \frac{1}{2\pi \sigma^2} e^\frac{-\left ( x^2 + y^2 \right )}{2\sigma^2} 
\end{equation}

em que o o filtro varia à escala através do parâmetro $ \sigma $

A função \textit{DoG} (\textit{"Diference of Gaussian"}) é dada pela diferença entre as imagens filtradas em escalas próximas separadas por uma constante \textit{k} e pode ser definida como:

\begin{equation}
DoG\left ( x, y, \sigma  \right ) = G\left ( x, y, k\sigma  \right ) * G\left ( x, y, \sigma \right )
\end{equation}

Assim, a convolução de uma imagem \textit{I} com o filtro \textit{DoG} é dado por:

\begin{equation}
D\left ( x, y, \sigma  \right ) = \left (  G\left ( x, y, k\sigma  \right ) - G\left ( x, y, \sigma \right ) \right )* I\left ( x, y \right ) = L\left ( x, y, k\sigma  \right ) - L\left ( x, y, \sigma  \right )
\end{equation}

que corresponde à diferença entre as imagens filtradas pelo filtro gaussiano em escalas $ \sigma $ e $ k\sigma $. Este filtro provoca uma perda de nitidez nas imagens (efeito desfoque) e torna-se assim capaz de detetar variações de intensidade nas imagens, como por exemplo, nos contornos. A Figura~\ref{fig:siftdog} mostra como é realizado o processo de obtenção das Diferenças Gaussianas e consequente formação das oitavas. 

\begin{figure}[h]
\centering
\includegraphics[width=0.7\linewidth]{./figures/sift_dog.jpg}
\caption{ Construção das Diferenças Gaussianas e formação de oitavas \textit{Retirada de}~\cite{Lowe2004} .}
\label{fig:siftdog}
\end{figure}

Segundo Lowe~\cite{Lowe2004}  é necessário atingir a escala $ 2\sigma $ para ser possível a construção de um descritor local invariável à escala, logo

$ k = 2^{\left(1/s \right)} $

onde s é o número de intervalos entre imagens obtidas por \textit{DoG} e $ D(x, y, \sigma) $ corresponde à primeira imagem e $ D(x, y, 2\sigma) $ à última de todo o conjunto de imagens geradas. Também deverão ser assim obtidas $s + 3$ imagens na pilha de imagens filtradas para cada oitava. Cada oitava contém as imagens de Diferença Gaussiana, sendo as que ficam entre as escalas superiores e inferiores designadas de intervalo. 

Por fim é realizado o processo de deteção de extremos, onde um pixel é comparado com os seus oito vizinhos na imagem atual e com os nove pixeis vizinhos das imagens de escalas adjacentes, numa região de 3x3. A figura~\ref{fig:siftdog2} ilustra este processo.  

\begin{figure}[h]
\centering
\includegraphics[width=0.4\linewidth]{./figures/sift_dog2.png}
\caption{Ilustração do processo de deteção de máximos e mínimos das imagens de Diferença Gaussiana. O pixel candidato está marcado com X e os vizinhos com um circulo. \textit{Retirada de}~\cite{Lowe2004}}
\label{fig:siftdog2}
\end{figure}

O próximo processo passa pela localização dos ponto chave.

\item \textbf{Localização de pontos chave}: quando detetado um extremo, esse ponto é considerado um candidato a ponto chave ou ponto de interesse. Este ponto foi encontrado através da comparação de um pixel com os seus vizinhos como referido anteriormente. sendo necessário realizar um cálculo de ajuste detalhado da localização e escala gaussiana de cada um destes pontos. É utilizada então a série de Taylor para obter uma localização mais exata dos extremos, sendo rejeitado caso a intensidade de um extremo seja inferior a um limiar previamente definido. 

Como DoG tem uma boa resposta a arestas e como estas fazem com que os pontos sejam instáveis com ruído, estes necessitam de ser removidos. É assim através da utilização de uma matriz Hessiana 2x2 possível calcular as curvaturas principais. Caso o rácio seja superior a um limiar previamente definido, o ponto é considerado uma aresta e assim o ponto chave é descartado.

Após a remoção de todos os pontos considerados não sendo de interesse, fica-se com todos os pontos chave, sendo necessário a atribuição das suas orientações. 

\item \textbf{Atribuição da orientação dos descritores}: este processo tem como principal finalidade possibilitar a representação de um descritor em relação a sua orientação, permitindo assim que este seja invariante a rotações. Para realizar esta tarefa é utilizada a escala Gaussiana $ \sigma $ para a escolha da imagem filtrada \textit{L} com a escala mais próxima e com a oitava referente ao ponto avaliado, tornando assim invariante também à escala.

Assim são calculados os gradientes para cada imagem $ L(x, y, \sigma) $ de intervalo, referentes às escalas e oitavas utilizadas.

A magnitude e orientação são calculados da seguinte forma:

\begin{equation}
m(x,y) = \sqrt{\left ( \left ( L(x+1, y) - L(x-1, y) \right )^2 + \left ( L(x, y+1) - L(x, y-1) \right )^2  \right )}
\end{equation}

\begin{equation}
\theta (x, y) = tan^{-1}\left ( \frac{\left ( L(x, y+1) - L(x, y-1) \right )}{\left ( L(x+1, y) - L(x-1, y) \right )} \right )
\end{equation}

Assim, é criado um histograma das orientações para pixeis numa região em redor do pontos chave, em que de todas as orientações obtidas para um ponto, apenas o maior pico e aquelas acima de 80\% do valor desse pico é que são utilizadas para definir a orientação de cada ponto chave.

Por fim, é possível a construção dos descritores para os pontos chave definidos como o ponto a seguir apresenta.

\item \textbf{Construção do descritor local}: este é o último passo em que é criado o descritor local para cada ponto de interesse. Para esse processo, é considerado um bloco 16x16 em redor do ponto chave e posteriormente dividido em 16 sub-blocos de 4x4. Por cada sub-bloco é criado um histograma com 8 picos relativos à orientação. Isto faz que no fim seja extraído um vetor de 128 posições para cada ponto chave. Para além deste retorno, são também tomadas várias medidas de modo a que exista robustez suficiente para que esse ponto de interesse seja invariante a mudanças de iluminação e rotação. 

\begin{figure}
\centering
\caption{Imagem}
% % falta imagem
\end{figure}
 
\end{enumerate}

O algoritmo SIFT é regularmente utilizado em reconhecimento de objetos ou cenas, tendo sido utilizado em deteção de objetos em \textit{frames} de vídeo~\cite{Sivic2003, Sivic2006} com utilização de técnicas mais avançadas de deteção de objetos e cenas utilizando vocabulários visuais como é apresentado na secção~\ref{subsec: vocab}.

%% SURF %% 
\subsubsection{SURF} \label{subsubsec:surf}

Outro algoritmo importante de extração de descritores locais é o SURF (\textit{Speeded Up Robust Features}). Este é baseado no SIFT apresentado na secção~\ref{subsubsec:sift} e também é utilizado, por exemplo, em reconhecimento de objetos ou mesmo na reconstrução 3D. Segundo os autores~\cite{Bay2006} o SURF é mais rápido (cerca de dez vezes) e robusto do que o SIFT, sendo que, segundo a comparação realizada em~\cite{Juan2009} comprovou-se que o SIFT é mais lento e não muito bom a mudanças de iluminação, mas apresenta melhores resultados a variações de rotação, mudanças de escala e transformações na imagem.

Este usa a técnica da imagem integral, onde cada pixel de uma imagem recebe um valor igual à soma dos pixeis da sua esquerda e acima, incluindo o próprio. Este também utiliza um filtro Haar em formato de caixa numa sub-região 4x4 em redor de um ponto de interesse, como se pode ver na Figura~\ref{fig:surf}, tornando o processo computacionalmente eficiente. Isto é realizado calculando a soma das respostas dos filtros e a soma do modulo das respostas dos filtros nas direções horizontal e vertical, gerando assim 4 valores por cada sub-região. Logo, o SURF retorna um vetor de 64 dimensões, metade do retornado pelo algoritmo SIFT.

\begin{figure}[h]
\centering
\includegraphics[width=0.7\linewidth]{./figures/surf}
\caption{ (a) Filtros Gaussianos de segunda ordem nas direções yy e xy; (b) aproximação por filtros de caixa; (c) filtros de Haar; As regiões a cinzento têm valor igual a zero. \textit{Retirada de}~\cite{Bay2006}.}
\label{fig:surf}
\end{figure}

%Para além destes dois algoritmos ainda existem outros menos utilizados como o GLOH (\textit{Gradient Location and Orientation Histogram}) e o HOG (\textit{Histogram of Oriented Gradients})~\cite{Dalal2005}.

%Na próxima secção e apresentado

\subsection{Descritores Baseado em Vocabulário Visual}\label{subsec: vocab}

No reconhecimento de documentos de texto é utilizado o conceito de vocabulário, sendo muitas vezes designado por \textit{BoW} (\textit{Bag Of Words}) que em português significa, saco de palavras, onde existe um conjunto de palavras armazenadas pré definidas. Um texto é assim caracterizado, através da análise das palavras que possui, sendo contabilizado a frequência de palavras que estejam simultaneamente no texto e no \textit{bag of words}, que assim atribuem um significado ao texto. É também utilizado uma lista de palavras que não acrescentam significado a expressões tais como, "o" ou "um", entre outras, que são removidas do texto durante a análise para não influenciarem os resultados. Um sistema de extração de informação de texto apresenta um número padrão de etapas~\cite{Baeza-Yates1999}. 

Recentemente esta técnica foi adotada em aplicações de extração de informação visual, como por exemplo é nos mostrado em~\cite{Sivic2003, Sivic2006}. Os autores recorrem a descritores locais invariantes a escala e rotação, para criar um vocabulário visual. A utilização de descritores locais como o SIFT, referido na secção~\ref{subsubsec:sift}, permite a extração de pontos de interesse nas imagens, invariantes a rotação e mudanças de escalas.

Quando efetuado este processo repetidamente com um grande conjunto de imagens, é possível criar \textit{clusters} de regiões de imagens muito semelhantes entre si, como se pode ver na Figura~\ref{fig:visualword}. Todas estas regiões são candidatas a palavras visuais, sendo selecionada a que representa melhor esse \textit{cluster}, isto é, é selecionado o centroide desse conjunto de regiões semelhantes entre si, recorrendo ao algoritmo \textit{k-means} referido na secção~\ref{subsec:parti}. Esse centroide passa então a ser considerado uma palavra visual e é adicionado a um vocabulário com outras palavras visuais.

\begin{figure}[h]
\centering
\includegraphics[width=0.4\linewidth]{./figures/visual_word_1}
\label{fig:visualword}
\end{figure}

Por fim é necessária a realização de uma indexação de cada palavra visual às imagens, sendo utilizado um vetor para cada imagem que indica, o número de vezes em que uma determinada palavra visual se repete numa imagem. Neste caso o vetor funciona como em \textit{text mining}, em que existe a contagem da frequência de palavras que ocorrem num determinado documento.

Outro processo possível para indexação do conteúdo visual ou texto, é atribuição de um peso ou ponderação para cada palavra visual numa determinada imagem. Aqui é utilizado a ponderação padrão conhecida como tf-idf ('\textit{term frequency-inverse document frequecy}')~\cite{Sivic2003}. 

Considerando um vetor com \textit{k} palavras visuais $Vd = (t_{1},...,t_{i},...,t_{k})$, em que a ponderação de cada palavra é dada pela equação~\ref{eq:peso}

\begin{equation}
t_{i} = \frac{n_{id}}{n_{d}}log\frac{N}{n_{i}}
\label{eq:peso}
\end{equation}

onde $n_{id}$ é o numero de ocorrências da palavra \textit{i} num documento \textit{d}, $n_{d}$ o número total de palavras no documento \textit{d}, $n_{i}$ é o numero de ocorrências da palavra \textit{i} em todos documentos e \textit{N} é o numero de documentos existentes. 

Conclui-se assim que este descritor permite a identificação de imagens semelhante ou reconhecimento de objetos, através da comparação dos vetores de cada imagem, com a informação relativa ao vocabulário posteriormente criado. Este demonstra ser bastante eficiente, e segundo Nistér e Stewénius~\cite{Nister2006} é possível escalar este processo para enormes quantidades de imagens sem perda de performance utilizando para isso um vocabulário em forma de árvore, isto é, com a hierarquização das palavras visuais de um vocabulário visual, recorrendo para isso ao \textit{clustering} hierárquico.

%Para a criação deste vocabulário visual, as regiões selecionadas podem ter em conta dois tipos diferentes de pontos de vista visual~\cite{Sivic2003, Sivic2006}. Um é designado por \textit{Shape Adapted} (SA), em que a forma adaptada sobre um ponto de interesse é elíptica. Este envolve um método iterativo para determinar o centro, escala e forma da elipse. O segundo, é designado por \textit{Maximally Stable} (MS), que tem como característica o facto de as regiões selecionadas serem as que a área é aproximadamente estacionária entanto o limiar de intensidade varia. Assim, as regiões SA tendem a centrar-se nos cantos e as regiões MS correspondem a formas de alto contraste em relação ao seu redor, como pode ser visualizado nos exemplos da figura~\ref{fig:visual_word}.
%
%Conclui-se assim que o objetivo é quantizar os descritores em \textit{clusters} que serão as "palavras" visuais. O quantização vetorial pode ser realizado com recurso ao algoritmo de \textit{clustering} \textit{k-means} referido na secção~\ref{subsec:parti}, ou também a histogramas~\cite{Sivic2003, Sivic2006}.

%Utilizado este tipo de descritor, é possível identificar imagens semelhantes, através da contagem de regiões de pontos-chave co-existentes entre as imagens. 

%\section{Trabalhos relacionados} \label{sec:work}
%
%A extração de informação em imagens é realizada já à vários anos, sendo muito utilizada na pesquisa de informação. Existem muitos trabalhos que, mesmo tendo objetivos diferentes, apresentam uma forte relação com o objetivo deste projeto de dissertação.
%Uma das áreas que se pode encontrar trabalhos relacionados é na deteção de objetos em vídeos~\cite{Sivic2003, Sivic2006}, onde se pretende identificar cenas de filmes semelhantes, tanto no mesmo filmes, como entre filmes diferentes, recorrendo a descritores baseados em vocabulário visual.
%Como é de esperar, também existe vários trabalhos relacionados, que apresentam métodos diferentes para encontrar imagens semelhantes, como na pesquisa web~\cite{Cai2004, Gao2005}.
%Na pesquisa bibliográfica realizada, foram encontrados dois trabalhos onde existe uma forte relação com o objetivo do projeto, e que apresentam técnicas que demonstraram bons resultados.
%O primeiro~\cite{Nister2006} utiliza os descritores baseados em vocabulário visual, tendo como diferenciação, a utilização de uma árvore hierárquica para reconhecimento de imagens semelhantes. O segundo
%~\cite{Lin2014}, apresenta conjuntos de imagens semelhantes através da utilização do algoritmo de \textit{clustering}, \textit{k-means}, sendo que recorre a uma modificação que torna o algoritmo mais rápido, reduzindo o tempo de processamento mesmo com uma grande base de dados. Para a realização deste este trabalho, foi utilizado histogramas de cor e histogramas preto e branco para descrever as imagens e ser possível assim identificar semelhanças entre elas.


\chapter{Plano de trabalho e conclusões}\label{chap:chap3}

Neste capítulo apresenta-se o plano de trabalho a ser realizado durante o desenvolvimento do projeto de dissertação através do diagrama de Gantt, com os períodos de tempo previstos para a sua execução, de acordo com os objetivos propostos no~\ref{chap:intro}, secção~\ref{sec:object}. São também referenciadas as ferramentas e tecnologias necessárias para o desenvolvimento deste projeto e uma pequena conclusão.

\section{Plano de trabalho}

Nesta secção é apresentado na figura~\ref{fig:gantt} o diagrama de Gantt com as tarefas previstas e com a respetiva distribuição das mesmas ao longo do tempo definido para o desenvolvimento do projeto de dissertação.  

\begin{figure}[h]
\scalebox{1}{
\begin{gantt}[xunitlength=0.5cm,fontsize=\small,titlefontsize=\small]{10}{18}
%\begin{gantt}{10}{18}
	\begin{ganttitle}
	    \titleelement{Feb}{3}
	    \titleelement{Mar}{4}
	    \titleelement{Apr}{5}
	    \titleelement{May}{4}
	    \titleelement{Jun}{2}
	\end{ganttitle}
	\begin{ganttitle}
	      \numtitle{2}{1}{4}{1}
	      \numtitle{1}{1}{4}{1}
	      \numtitle{1}{1}{5}{1}
	      \numtitle{1}{1}{4}{1}
	      \numtitle{1}{1}{2}{1}
	    \end{ganttitle}
	\ganttbar[color=red]{Implemetação sistemas acesso às imagens}{0}{2}
	\ganttbar[color=red]{Familiarização com ferramentas de trabalho }{1}{2}
	\ganttbar[color=red]{Especificação do sistema a implementar}{2}{2}
	\ganttbar[color=orange]{Desenvolvimento da primeira versão do modelo}{4}{3}
	\ganttbarcon[color=orange]{Desenvolvimento da segunda versão do modelo}{7}{4}
	\ganttbar[color=yellow]{Extensão do TweeProfiles}{11}{4}
	\ganttbar[color=green]{Escrita da Dissertação}{14}{4}
	\ganttbar[color=cyan]{Criação e actualização do website}{0}{18}

\end{gantt}
}
\caption{Diagarama de Gantt com plano de trabalho}
\label{fig:gantt}
\end{figure}

\begin{description}
\item[Implementação sistemas acesso ás imagens:] Para a realização deste projeto será necessário realizar a recolha dos \textit{urls} das imagens partilhadas através do serviço Twitter e o armazenamento das respetivas imagens para que possam posteriormente ser utilizadas para o objetivo desta dissertação referido no capítulo~\ref{chap:intro}, secção~\ref{sec:object}.A duração prevista para esta etapa é de \textbf{duas semanas}.

\item[Familiarização com ferramentas de trabalho:] Nesta fase inicial, é necessário fazer uma avaliação e o estudo das ferramentas e do ambiente de desenvolvimento para as fases seguinte. A duração prevista para esta etapa é de \textbf{duas semana}.

\item[Especificação do sistema a implementar: ] Em paralelo com a etapa anterior, dá-se início à especificação do sistema que irá analisar as imagens e ser capaz de apresentar \textit{clusters} das mesmas. Será também definida a estratégia de ataque ao problema. A duração prevista para esta etapa é de cerca de \textbf{duas semanas}.

\item[Desenvolvimento do sistema :] Esta etapa encontra-se dividida em duas sub-etapas para a implementação do sistema a desenvolver. Enquanto que a segunda  Esta etapa terá uma duração total prevista de \textbf{sete semanas}

\begin{enumerate}
\item \textbf{Primeira versão: } Que seja capaz de realizar a tarefa de \textit{clustering} recorrendo a apenas algumas imagens. Esta Sub-tarefa terá uma duração prevista de \textbf{três semanas}
\item \textbf{Segunda versão: } Que seja capaz de realizar a tarefa de \textit{clustering} recorrendo a uma base de dados maior. Esta Sub-tarefa terá uma duração prevista de \textbf{quatro semanas}
\end{enumerate}

\item[Extensão do TweeProfiles: ] Esta etapa refere-se ao processo de implementação do sistema desenvolvido na etapa anterior, em que se procederá à extensão da ferramenta TweeProfiles~\citet{Cunha2013} com teste. Esta etapa tem uma duração prevista de cerca de \textbf{quatro semanas}.  

\item[Escrita da Dissertação: ] Após todos os testes realizados e feita a análise dos resultados obtidos, será realizado a última etapa, que corresponde à escrita do documento final. que deverá descrever todo o trabalho desenvolvido ao longo do semestre e as conclusões retiradas. Esta etapa tem uma duração prevista de cerca de \textbf{quatro semanas}.

\item[Criação e atualização da página web: ] Esta será uma tarefa continua ao longo do desenvolvimento da dissertação, em que será criada uma pagina web com informação relativa ao projeto, e que será constantemente atualizada com os respetivos desenvolvimentos.
\end{description}


\section{Ferramentas de apoio}

Nesta secção faz-se referência às ferramentas e tecnologias utilizadas para o desenvolvimento do trabalho.

Todos os tweets estão alojados numa base de dados Mongodb. Para acesso à informação será utilizada a linguagem Python, mais especificamente, através da utilização da biblioteca Pymongo. Será também necessário recorrer a uma base de dados relacional PostgreSQL para o armazenamento dos endereços de acesso à fonte das fotografias e de outras informações pertinentes que permita uma pesquisa facilmente filtrada. 

%Todos os tweets estão alojados numa base de dados Mongodb, sendo possível recorrer à informação armazenada recorrendo à linguagem Python, mais precisamente com a biblioteca Pymongo. Será também necessário a utilização de uma base de dados relacional PostgreSQL para o armazenamento dos endereços à fonte das fotografias e de outras informações pertinentes que facilite a filtragem das mesmas. 

Para o desenvolvimento do sistema de processamento das imagens será utilizado a ferramenta livre OpenCV.


\section{Conclusão}

Este relatório foi realizado com o intuito de preparar a realização do projeto de dissertação. Assim, primeiramente é apresentada uma introdução com a descrição do tema, das motivações e objetivos deste trabalho. De seguida, o capítulo~\ref{chap:estarte} apresenta o estado da arte, onde foi realizada a revisão bibliográfica, recorrendo a pesquisa de livros e artigos científicos, informação esta que se revelou importante para a compreensão das tarefas necessárias a realizar, bem como dos conteúdos a dominar. Para além disto, este capitulo faz uma introdução teórica a alguns conteúdos que são necessários dominar. Por fim, foi apresentado o diagrama de Gantt que descreve as tarefas a realizar e a correspondente duração, e ainda, as ferramentas necessárias para o desenvolvimento do projeto.


%Este relatório foi realizado com o intuito de preparar a realização do projeto de dissertação. Assim, o primeiro passo foi a apresentação de uma breve introdução com a descrição do tema, das motivações e objetivos deste trabalho. Já o capítulo~\ref{chap:estarte}, apresenta o estado da arte, onde foi realizada a revisão bibliográfica, recorrendo a pesquisa de livros e artigos científicos, informação esta que se revelou importante para a compreensão das tarefas necessárias a realizar, bem como dos conteúdos a dominar. Para além disto, este capitulo faz uma introdução teórica a alguns conteúdos que são necessários dominar.Por fim, foi apresentado o diagrama de Gantt que descreve as tarefas a realizar e a correspondente duração e as ferramentas que serão utilizadas para o desenvolvimento do projeto.

Após a finalização deste relatório final, conclui-se que este projeto possui suporte cientifico e tecnológico para que seja concretizado. 



% % % % % % % % % % % % % % % % % % % % % % % % % % % % % % % % % % % % % % % % % % % % %
%Apresenta-se de seguida um exemplo de equação, completamente fora do contexto:
%\begin{eqnarray}
%CIF_1: \hspace*{5mm}F_0^j(a) &=& \frac{1}{2\pi \iota} \oint_{\gamma} \frac{F_0^j(z)}{z - a} dz\\
%CIF_2: \hspace*{5mm}F_1^j(a) &=& \frac{1}{2\pi \iota} \oint_{\gamma} \frac{F_0^j(x)}{x - a} dx \label{eq:cif}
%\end{eqnarray}
%
%Na Equação~\ref{eq:cif} lorem ipsum dolor sit amet, consectetuer
%adipiscing elit. Suspendisse tincidunt viverra elit. Donec tempus
%vulputate mauris. Donec arcu. Vestibulum condimentum porta
%justo. Curabitur ornare tincidunt lacus. Curabitur ac massa vel ante
%tincidunt placerat. Cras vehicula semper elit. Curabitur gravida, est
%a elementum suscipit, est eros ullamcorper quam, sed cursus velit
%velit tempor neque. Duis tempor condimentum ante. Nam
%sollicitudin. Vestibulum adipiscing, orci eu tempor dapibus, risus
%sapien porta metus, et cursus leo metus eget nibh. 
%
%\section{Secção Exemplo}
%
%A arquitectura do visualizador assenta sobre os seguintes conceitos
%base~\citep{kn:ZPMD97}: 
%
%\begin{itemize}
%\item \textbf{Componentes} --- Suspendisse auctor mattis augue \emph{push};
%\item \textbf{Praesent} --- Sit amet sem maecenas eleifend facilisis leo;
%\item \textbf{Pellentesque} --- Habitant morbi tristique senectus et netus.
%\end{itemize}
%
%\subsection{Exemplo de Figura}
%
%É apresentado na Figura~\ref{fig:arch} da página~\pageref{fig:arch} um
%exemplo de figura flutuante que deverá ficar no topo da página.
%
%\begin{figure}[t]
%  \begin{center}
%    \leavevmode
%    \includegraphics[width=0.86\textwidth]{puzzle}
%    \caption{Arquitectura da Solução Proposta}
%    \label{fig:arch}
%  \end{center}
%\end{figure}
%
%Loren ipsum dolor sit amet, consectetuer adipiscing elit. 
%Praesent sit amet sem. Maecenas eleifend facilisis leo. Vestibulum et
%mi. Aliquam posuere, ante non tristique consectetuer, dui elit
%scelerisque augue, eu vehicula nibh nisi ac est. Suspendisse elementum
%sodales felis. Nullam laoreet fermentum urna. 
%
%
%\subsection{Exemplo de Tabela}
%
%É apresentado na Tabela~\ref{tab:exemplo} um exemplo de tabela
%flutuante que deverá ficar no topo da página.
%
%\begin{table}[t]
%  \centering
%  \caption{Tabela Exemplo}
%\begin{tabular}{|c|r@{.}lr@{.}lr@{.}l||r|}
%	\hline
%\multicolumn{8}{|c|}
%	{\rule[-3mm]{0mm}{8mm}Iteração $k$ de $f(x_n)$} \\
%\textbf{\em k}
%	& \multicolumn{2}{c}{$x_1^k$}
%	& \multicolumn{2}{c}{$x_2^k$}
%	& \multicolumn{2}{c||}{$x_3^k$}
%	& comentários \\ \hline \hline
%0   & -0&3                 & 0&6                 &  0&7   & - \\
%1   &  0&47102965 & 0&04883157 & -0&53345964  & $\delta<\epsilon$ \\
%2   &  0&49988691 & 0&00228830 & -0&52246185  & $\delta < \varepsilon$ \\
%3   &  0&49999976 & 0&00005380 & -0&523656   &   $N$ \\
%4   &  0&5                 & 0&00000307 & -0&52359743  & \\
%\vdots	& \multicolumn{2}{c}{\vdots}
%	& \multicolumn{2}{c}{$\ddots$}
%	& \multicolumn{2}{c||}{\vdots}  & \\
%7   &  0&5   & 0&0    & \textbf{-0}&\textbf{52359878}
%		 & $\delta<10^{-8}$ \\ \hline
%\end{tabular}
%  \label{tab:exemplo}
%\end{table}
%


\chapter{Olhó-passarinho}\label{chap:chap4}

Neste capitulo será abordado a ferramenta desenvolvida com a descrição da arquitetura sistema implementado, do processamento da informação espaço-temporal e da sua integração com a informação visual de modo a aplicar a tarefa de \textit{clustering}. Por fim será apresentado a visualização dos resultados ilustrativos e consequentemente a sua discussão. 

\section{Arquitetura do sistema}

A arquitetura do sistema desenvolvido é apresentado na figura~\ref{fig:archsys}. Este apresenta uma divisão entre os serviços externos e o modelo desenvolvido. Este modelo foi desenhado de modo a que existisse uma separação entre o tratamento de toda a parte de processamento dos dados e a visualização, existindo assim um \textit{back-end} com todos os ficheiros e módulos desenvolvidos e um \textit{front-end} que representa a aplicação web para visualização dos resultados. No \textit{back-end} existe também uma divisão entre dois módulos fundamentais, o módulo de processamento da informação visual, responsável por tratar a informação das imagens como descrito no Capítulo~\ref{chap:chap3}, de modo a que essa informação possa ser utilizada pelo módulo responsável pelo processo de \textit{Data Mining} já desenvolvido no TweeProfiles~\cite{Cunha2013}. 

Os serviços externos correspondem à base de dados Mongodb para a recolha dos tweets e os serviços Twitter e Instagram para a recolha das imagens através do URL. No caso do modelo desenvolvido, a parte de \textit{back-end} possuí os ficheiros JSON com os dados e as imagens necessárias, tanto para o módulo de processamento da informação visual como para a extração e processamento do dados espaço-temporais, explicados na próxima secção~\ref{sec:infoesptmp}. Os dados processados no módulo da informação visual e os dados espaço-temporal extraídos dos tweets são assim utilizados no processo de \textit{Data Mining}, onde é aplicada a tarefa de \textit{clustering} como explicado na secção~\ref{sec:finalclustering} apresentada mais adiante. De este processo resultam os \textit{clusters} calculados através de vários parâmetros, sendo estas informações armazenadas em ficheiros. Por fim, foi utilizada a \textit{microframework} Flask para desenvolvimento de aplicações web em Python, que permitiu o desenvolvimento da aplicação Olhó-passarinho para visualização dos resultados. 

\begin{figure}[h]
\centering
\includegraphics[width=1.0\linewidth]{./figures/arquitetura_sistema}
\caption{Arquitetura do sistema completo}
\label{fig:archsys}
\end{figure}


\section{Informação espaço-temporal} \label{sec:infoesptmp}

Uma das características principais tanto do TweeProfiles como do Olhó-paddarinho e é integração das dimensões espaço-temporais com o conteúdo. Tal como no foi efetuado no TweeProfiles, também aqui foi utilizado estas dimensões, tendo sido então recolhido a informação de tempo e espaço dos tweets para o cálculo das respetivas matrizes de distância entre tweets.

Em primeiro lugar foi recolhida a informação espacial. Neste caso os dados possuem a informação de latitude e longitude do ponto onde foi enviado o tweet. Para calcular a distância entre tweets utilizou-se a função distância Haversine abordada no capítulo~\ref{chap:estarte} na subsecção~\ref{subsubsec:space}. Neste caso foi calculada a distância em quilómetros, tendo sido considerado o valor do raio da Terra igual a 6371 Km.

Posteriormente foi então recolhida a informação temporal dos tweets. Esta informação apresenta-se no seguinte formato:

\vspace{2mm}
\centering\textbf{Tue Jun 18 17:02:09 +0000 2013}

Para o cálculo da distância entre data foi utilizada a distância a função distância euclidiana...


%\section{Clustering da informação visual, espacial e temporal} \label{sec:finalclustering}

%\section{Visualização}

%\section{Resultados ilustrativos}
\chapter{Conclusões e Trabalho Futuro} \label{chap:concl}

Neste capítulo são expostas algumas conclusões retiradas do desenvolvimento desta dissertação e são apresentadas sugestões para um trabalho futuro com indicação de melhorias a implementar e sugestões de 

\section{Resumo do Trabalho Realizado}

Durante o período dedicado à realização do projeto de dissertação, foram seguidas uma sequência definida de etapas que culminou num sistema capaz de reproduzir a visualização no espaço, tempo de \textit{clusters} e visualizar e navegar por fotografias partilhadas no serviço de \textit{microblogging} Twitter contidas num determinado \textit{cluster}.

Inicialmente foi feita uma recolha dos dados necessários ao desenvolvimento deste projeto de dissertação. Este dados foram recolhidos através de base de dados Mongodb e possuíam a informação relativa a tweets partilhados na rede social Twitter. Como o objetivo era a descoberta de padrões através de fotografias, foi deito o \textit{download} de todas  as imagens pertencentes a tweets e partilhadas no Twitter através do serviço Instagram. Os dados desses tweets também foram armazenados no formato JSON.

O próximo passo foi o desenvolvimento de um módulo responsável pela extração, processamento e armazenamento da informação visual. Este foi desenvolvido para que representasse as imagens de uma forma mais eficiente e compacta, e que tornasse assim possível a comparação entre imagens para a criação de uma matriz distância para ser utilizada no processo de \textit{Data Mining}

Prossegui-se com a produção das matrizes de distância entre tweets pelas dimensões temporais, de forma a combinar esta informação com a informação visual, as fotografias. Com esta integração concluída utilizou-se essa informação no processo de \textit{Data Mining} para a obtenção dos \textit{clusters}, com a atribuição de diferentes pesos às diferentes dimensões. 

Após a obtenção dos diferentes \textit{clusters}, foi desenvolvida a aplicação web em Python, recorrendo a microframework Flask, para visualização dos resultados através do conteúdo dos tweets e das diferentes dimensões já referidas, resultando assim num sistema completo e funcional.


\section{Trabalho Futuro}

Após a finalização deste projeto de dissertação foi feita uma análise a todo o processo realizado, tendo sido concluído que os objetivos principais propostos foram atingidos. Apesar disso, alguns objetivos mais ambiciosos não foram possíveis ser atingidos devido a vários fatores, e que devem ser tidos em conta num trabalho futuro.

Um dos pontos de partida que num trabalho futuro deve ser tido em conta é a possibilidade de aceder a uma base de dados maior, pois apesar de existirem muitas imagens partilhadas no Twitter através de vários serviços, o número de tweets georeferenciados ainda é bastante reduzido. Para além disso, era interessante utilizar uma base de dados com uma extensão temporal superior e consequentemente, com conteúdos mais diversificados.

O desenvolvimento do modelo responsável pelo tratamento da informação visual, mais concretamente, da criação de um vocabulário visual, apresentou-se como uma boa opção com resultados muito satisfatórios, mas num trabalho futuro também seria interessante a integração de outros descritores, como por exemplo, descritores de cor, adicionando assim a componente cor Isto iria permitir uma melhor descrição das imagens e possibilitaria identificar cenários mais específicos onde a cor é um fator determinante de distinção, como por exemplo, fotografias de praias, alimentos ou mesmo locais com vegetação, como jardins ou parques naturais onde predomina a cor verde. 

Já na parte da aplicação, existe diferentes abordagens a poderem ser seguidas para diferenciarem a visualização dos resultados, e consequentemente melhorarem a capacidade do utilizador compreender melhor o que está a visualizar. Uma das possibilidades, seria a síntese de uma imagem que fosse representativa do \textit{cluster} a que pertence, isto é, um sistema que analisasse todas as imagens contidas num \textit{cluster}, e fosse capaz de reproduzir uma imagem modelo, através da informação de todas as imagens do \textit{cluster}, e até mesmo, através de informação existente numa base de dados de imagens exterior.

Por fim, a utilização de uma ferramenta com este intuito tornar-se-ia mais interessante se, a informação disponível para visualização fosse constantemente atualizada. Para isso seria necessário a utilização de hardware com capacidade suficiente de analisar as imagens em tempo real e exportar a informação necessária ao processo de \textit{Data Mining}. A utilização do vocabulário visual iria permitir a utilização de um serviço assim, sendo que seria necessário realizar algumas alterações, como por exemplo utilizar um vocabulário visual disposto de forma hierárquica, o que permitiria uma mais rápida descrição de uma imagem. Para além disso, o processo de \textit{Data Mining} teria de estar constantemente em funcionamento de forma a atualizar os \textit{clusters} sempre que existissem alterações nos mesmos ou mesmo no aparecimento de novos.

 

%% Comment next 2 commands if numbered appendices are not used
%\appendix
\appendix
\chapter{Tabela da Base de Dados para Seleção de Tweets} \label{ap1}


\begin{lstlisting}[language=SQL]
create table if not exists IMAGENS ( 
	id integer PRIMARY KEY AUTOINCREMENT, 
	id_tweet text, 
	servico text, 
	url text, 
	tipo text, 
	retweet text 
); 
\end{lstlisting}




%%----------------------------------------
%% Final materials
%%----------------------------------------

%% Bibliography
%% Comment the next command if BibTeX file not used, 
%% Assumes that bibliography is in ``myrefs.bib''
\PrintBib{myrefs}

%% Index
%% Uncomment next command if index is required, 
%% don't forget to run ``makeindex tese'' command
%\PrintIndex

\end{document}
