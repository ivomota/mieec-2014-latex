\chapter{Conclusões e Trabalho Futuro} \label{chap:concl}

Neste capítulo são expostas algumas conclusões retiradas do desenvolvimento desta dissertação e são apresentadas sugestões para um trabalho futuro com indicação de melhorias a implementar e sugestões de 

\section{Resumo do Trabalho Realizado}

Durante o período dedicado à realização do projeto de dissertação, foram seguidas uma sequência definida de etapas que culminou num sistema capaz de reproduzir a visualização no espaço, tempo de \textit{clusters} e visualizar e navegar por fotografias partilhadas no serviço de \textit{microblogging} Twitter contidas num determinado \textit{cluster}.

Inicialmente foi feita uma recolha dos dados necessários ao desenvolvimento deste projeto de dissertação. Este dados foram recolhidos através de base de dados Mongodb e possuíam a informação relativa a tweets partilhados na rede social Twitter. Como o objetivo era a descoberta de padrões através de fotografias, foi deito o \textit{download} de todas  as imagens pertencentes a tweets e partilhadas no Twitter através do serviço Instagram. Os dados desses tweets também foram armazenados no formato JSON.

O próximo passo foi o desenvolvimento de um módulo responsável pela extração, processamento e armazenamento da informação visual. Este foi desenvolvido para que representasse as imagens de uma forma mais eficiente e compacta, e que tornasse assim possível a comparação entre imagens para a criação de uma matriz distância para ser utilizada no processo de \textit{Data Mining}

Prossegui-se com a produção das matrizes de distância entre tweets pelas dimensões temporais, de forma a combinar esta informação com a informação visual, as fotografias. Com esta integração concluída utilizou-se essa informação no processo de \textit{Data Mining} para a obtenção dos \textit{clusters}, com a atribuição de diferentes pesos às diferentes dimensões. 

Após a obtenção dos diferentes \textit{clusters}, foi desenvolvida a aplicação web em Python, recorrendo a microframework Flask, para visualização dos resultados através do conteúdo dos tweets e das diferentes dimensões já referidas, resultando assim num sistema completo e funcional.


\section{Trabalho Futuro}

Após a finalização deste projeto de dissertação foi feita uma análise a todo o processo realizado, tendo sido concluído que os objetivos principais propostos foram atingidos. Apesar disso, alguns objetivos mais ambiciosos não foram possíveis ser atingidos devido a vários fatores, e que devem ser tidos em conta num trabalho futuro.

Um dos pontos de partida que num trabalho futuro deve ser tido em conta é a possibilidade de aceder a uma base de dados maior, pois apesar de existirem muitas imagens partilhadas no Twitter através de vários serviços, o número de tweets georeferenciados ainda é bastante reduzido. Para além disso, era interessante utilizar uma base de dados com uma extensão temporal superior e consequentemente, com conteúdos mais diversificados.

O desenvolvimento do modelo responsável pelo tratamento da informação visual, mais concretamente, da criação de um vocabulário visual, apresentou-se como uma boa opção com resultados muito satisfatórios, mas num trabalho futuro também seria interessante a integração de outros descritores, como por exemplo, descritores de cor, adicionando assim a componente cor Isto iria permitir uma melhor descrição das imagens e possibilitaria identificar cenários mais específicos onde a cor é um fator determinante de distinção, como por exemplo, fotografias de praias, alimentos ou mesmo locais com vegetação, como jardins ou parques naturais onde predomina a cor verde. 

Já na parte da aplicação, existe diferentes abordagens a poderem ser seguidas para diferenciarem a visualização dos resultados, e consequentemente melhorarem a capacidade do utilizador compreender melhor o que está a visualizar. Uma das possibilidades, seria a síntese de uma imagem que fosse representativa do \textit{cluster} a que pertence, isto é, um sistema que analisasse todas as imagens contidas num \textit{cluster}, e fosse capaz de reproduzir uma imagem modelo, através da informação de todas as imagens do \textit{cluster}, e até mesmo, através de informação existente numa base de dados de imagens exterior.

Por fim, a utilização de uma ferramenta com este intuito tornar-se-ia mais interessante se, a informação disponível para visualização fosse constantemente atualizada. Para isso seria necessário a utilização de hardware com capacidade suficiente de analisar as imagens em tempo real e exportar a informação necessária ao processo de \textit{Data Mining}. A utilização do vocabulário visual iria permitir a utilização de um serviço assim, sendo que seria necessário realizar algumas alterações, como por exemplo utilizar um vocabulário visual disposto de forma hierárquica, o que permitiria uma mais rápida descrição de uma imagem. Para além disso, o processo de \textit{Data Mining} teria de estar constantemente em funcionamento de forma a atualizar os \textit{clusters} sempre que existissem alterações nos mesmos ou mesmo no aparecimento de novos.

